\documentclass{llncs}

\usepackage[utf8]{inputenc}
\usepackage[T1]{fontenc}
\usepackage{lmodern}
\usepackage[ngerman]{babel}

\usepackage[backend=biber]{biblatex}
%\bibliographystyle{splncs}
\bibliography{../parallele-algorithmen} 

\begin{document}

\pagestyle{headings}

\title{Parallele Algorithmen}

\author{Lennart Braun \and Francisco CardosoLennart Braun}

\institute{
    Universität Hamburg,
    Fakultät für Mathematik, Informatik und Naturwissenschaften,
    Fachbereich Informatik,
    Arbeitsbereich TGI,
    Proseminar Nebenläufigkeit SS 14
    \email{3braun@informatik.uni-hamburg.de} \\
    \email{3cardoso@informatik.uni-hamburg.de}
}

\maketitle

\begin{abstract}
Wir stellen verschiedene Modelle für parallele Architekturen vor und gehen auf die verschiedenen Eigenschaften ein. Weiterhin wird auf Möglichkeiten eingegangen, Algorithmen zu paralelisieren.
\keywords{parallele Algorithmen, PRAM, Sortieralgorithmen}
\end{abstract}

\section{Einleitung}

\subsection{Motivation}
Computer werden immer leistungsfähiger.
Bis zur Mitte des ersten Jahrzehnts nach der Jahrtausendwende stieg der
Prozessortakt kontinuierlich an und die Ausführung einzelner Instruktionen
wurde optimiert.
Allerdings stagniert der Trend seitdem.
Es sind physikalische Limits erreicht; ein weiterer Anstieg kann nur noch
mit hohem Energieverbrauch und viel Abwärme erkauft werden.
Der Nutzen wird einschränkt.
Die Aussagen bedeuten jedoch nicht, dass moderne Prozessoren nicht effizienter
sein können.
Die Multicore-Architekturen setzen mehrere Prozessoren auf einem Chip ein,
zudem wird Hyper-Threading standardmäßig genutzt.
Aber ein Programm läuft nicht von allein viermal so schnell, wenn man es auf
einer Maschine mit Dualcore Prozessor einsetzt.
Im Gegensatz zur Erhöhung der Taktfrequenz, kann Software nicht einfach mehrere
Kerne verwenden.
Wie Herb Sutter in seinem Artikel „The Free Lunch Is Over“
\cite{sutterlunch} beschreibt, ist es heute notwendig, die Programme anzupassen
und es auszunutzen, dass nebenläufige Ausführung möglich ist.
Man kann nicht mehr darauf vertrauen, dass die Software mit jeder Generation
von Prozessoren schneller läuft.
Die Algorithmen müssen angepasst und parallelisiert werden, falls möglich, um
eine effiziente Auslastung von modernen Computer zu gewährleisten.

Diese Ausarbeitung soll einen groben Überblick über parallele Algorithmen geben.
Es werden theoretische Modelle vorgestellt, anhand derer Algorithmen analysiert
werden können, und es wird auf die Frage eingegangen, welche Probleme überhaupt
parallel gelöst werden können.

\subsection{Der Begriff Parallelität}

\subsection{Aufbau}
In den ersten beiden Abschnitten werden abstrakte Modelle vorgestellt, anhand
derer sequentielle und parallele Algorithmen betrachtet werden können.
Kapitel 4 behandelt die Komplexitätsklassen der beobachteten Algorithmen.
Es wird darauf eingegangen, welche Probleme sich überhaupt effizient parallel
gelöst werden können.

Im Laufe des Textes werden verschiedene Angaben für Komplexität verwendet.
Dabei wird ein uniformes Komplexitätsmaß verwendet.
Die Menge an verwendeten Variablen, in denen Werte während der Ausführung
zwischengespeichert werden, ist die Platzkomplexität ($space$).
Die Zeitkomplexität ($time$) eines Algorithmus gibt die Anzahl der
Zeiteinheiten an, die zur Ausführung gebraucht werden.
Eine Anweisung benötigt eine Zeiteinheit.
Allerdings können insbesondere bei parallelen Algorithmen Anweisungen parallel
ausgeführt werden.
Die Zeitkomplexität entspricht daher nicht unbedingt der Anzahl ausgeführter
Anweisungen ($work$).
Diese wird auch als Operationenkomplexität bezeichnet.
Des weiteren wird bei parallelen Algorithmen die Anzahl der verwendeten
Prozessoren betrachtet.

\section{Sequentielle Modelle}

%----------------------------------------------------------------------%SLIDE -
\begin{frame}
    \frametitle{Sequentielle Modelle}
    \begin{itemize}
        \item gleich Mächtig zu Turingmaschine
        \item einfacher zu benutzen
        \item Basis für parallele Modelle
    \end{itemize}
    SLP und RAM
\end{frame}
%----------------------------------------------------------------------%SLIDE -

\subsection{Straight Line Program}
%----------------------------------------------------------------------%SLIDE -
\begin{frame}
    \frametitle{Straight Line Program}
    \begin{itemize}
        \item Beschreibung
    \end{itemize}
\end{frame}
%----------------------------------------------------------------------%SLIDE -

\subsection{Random Access Machine}
%----------------------------------------------------------------------%SLIDE -
\begin{frame}
    \frametitle{Random Access Machine}
    \begin{itemize}
        \item Beschreibung
    \end{itemize}
\end{frame}
%----------------------------------------------------------------------%SLIDE -

\section{Parallele Modelle}

%----------------------------------------------------------------------%SLIDE -
\begin{frame}
    \frametitle{Parallele Modelle}
    Circuits, Netzwerke, PRAMs
\end{frame}
%----------------------------------------------------------------------%SLIDE -


\subsection{Circuit}
%----------------------------------------------------------------------%SLIDE -
\begin{frame}
    \frametitle{Circuit}
    \begin{columns}
        \begin{column}{0.5\textwidth}
            \begin{itemize}
                \item gerichteter, azyklischer Graph (DAG)
                \item Eingabe, Ausgabe
                \item Operationen
                \item Boolean Circuit
                \item interessant für theoretische Überlegungen
            \end{itemize}
        \end{column}
        \begin{column}{0.5\textwidth}
            \begin{figure}
                \centering
                % sequential circuit computing the sum of eight numbers
\begin{tikzpicture}
    [
        vertex/.style={circle, scale=1.2, draw,},
        edge/.style={-latex,},
        scale=1.5,
    ]

    % sequential dag
    \node (1) at (0,0)    [vertex] {};
    \node (2) at (0,1)    [vertex] {};
    \node (3) at (0,2)    [vertex] {};
    \node (4) at (1,0)    [vertex] {};
    \node (5) at (1.5,1)    [vertex] {};
    \node (6) at (2,0)    [vertex] {};
    \node (7) at (3,0)    [vertex] {};
    \node (8) at (3,2)    [vertex] {};

    \draw [edge] (1) -- (4);
    \draw [edge] (2) -- (4);
    \draw [edge] (2) -- (5);
    \draw [edge] (2) -- (8);
    \draw [edge] (3) -- (5);
    \draw [edge] (4) -- (6);
    \draw [edge] (5) -- (6);
    \draw [edge] (6) -- (7);
    \draw [edge] (6) -- (8);

\end{tikzpicture}         

                \caption{Ein DAG}
                \label{fig:dag}
            \end{figure}
        \end{column}
    \end{columns}
\end{frame}
%----------------------------------------------------------------------%SLIDE -

%----------------------------------------------------------------------%SLIDE -
\begin{frame}[b]
    \frametitle{Boolean Circuit}
    \framesubtitle{Addition von acht Zahlen}
    \begin{columns}[b]
        \column{0.5\textwidth}
        \begin{figure}
            \centering
            % sequential circuit computing the sum of eight numbers
\begin{tikzpicture}
    [
        vertex/.style={circle, draw, scale=0.55, minimum size=3ex,},
        edge/.style={-latex,},
        scale=0.65,
    ]

    % sequential dag
    \node (b1) at (-3,0)    [vertex] {$A_1$};
    \node (b2) at (-2,0)    [vertex] {$A_2$};

    \node (q1) at (-2.5,1)  [vertex] {$+$};
    \node (b3) at (-1.5,1)  [vertex] {$A_3$};

    \node (q2) at (-2,2)    [vertex] {$+$};
    \node (b4) at (-1,2)    [vertex] {$A_4$};

    \node (q3) at (-1.5,3)  [vertex] {$+$};
    \node (b5) at (-0.5,3)  [vertex] {$A_5$};

    \node (q4) at (-1,4)    [vertex] {$+$};
    \node (b6) at (0,4)     [vertex] {$A_6$};

    \node (q5) at (-0.5,5)  [vertex] {$+$};
    \node (b7) at (0.5,5)   [vertex] {$A_7$};

    \node (q6) at (0,6)     [vertex] {$+$};
    \node (b8) at (1,6)     [vertex] {$A_8$};

    \node (q7) at (0.5,7)   [vertex] {$+$};

    \draw [edge] (b1) -- (q1);
    \draw [edge] (b2) -- (q1);
    \draw [edge] (b3) -- (q2);
    \draw [edge] (b4) -- (q3);
    \draw [edge] (b5) -- (q4);
    \draw [edge] (b6) -- (q5);
    \draw [edge] (b7) -- (q6);
    \draw [edge] (b8) -- (q7);
    \draw [edge] (q1) -- (q2);
    \draw [edge] (q2) -- (q3);
    \draw [edge] (q3) -- (q4);
    \draw [edge] (q4) -- (q5);
    \draw [edge] (q5) -- (q6);
    \draw [edge] (q6) -- (q7);

\end{tikzpicture}         

            \caption{eine Möglichkeit}
        \end{figure}
        \pause
        \column{0.5\textwidth}
        \begin{figure}
            \centering
            % parallel circuit computing the sum of eight numbers
\begin{tikzpicture}
    [
        vertex/.style={circle, draw, scale=0.55, minimum size=3ex,},
        edge/.style={-latex,},
        scale=0.65,
    ]

    % parallel dag
    \node (a1) at (0,0)     [vertex] {$A_1$};
    \node (a2) at (1,0)     [vertex] {$A_2$};
    \node (a3) at (2,0)     [vertex] {$A_3$};
    \node (a4) at (3,0)     [vertex] {$A_4$};
    \node (a5) at (4,0)     [vertex] {$A_5$};
    \node (a6) at (5,0)     [vertex] {$A_6$};
    \node (a7) at (6,0)     [vertex] {$A_7$};
    \node (a8) at (7,0)     [vertex] {$A_8$};
    \node (p1) at (0.5,1.5) [vertex] {$+$};
    \node (p2) at (2.5,1.5) [vertex] {$+$};
    \node (p3) at (4.5,1.5) [vertex] {$+$};
    \node (p4) at (6.5,1.5) [vertex] {$+$};
    \node (p5) at (1.5,3)   [vertex] {$+$};
    \node (p6) at (5.5,3)   [vertex] {$+$};
    \node (p7) at (3.5,4.5) [vertex] {$+$};
    \draw [edge] (a1) -- (p1);
    \draw [edge] (a2) -- (p1);
    \draw [edge] (a3) -- (p2);
    \draw [edge] (a4) -- (p2);
    \draw [edge] (a5) -- (p3);
    \draw [edge] (a6) -- (p3);
    \draw [edge] (a7) -- (p4);
    \draw [edge] (a8) -- (p4);
    \draw [edge] (p1) -- (p5);
    \draw [edge] (p2) -- (p5);
    \draw [edge] (p3) -- (p6);
    \draw [edge] (p4) -- (p6);
    \draw [edge] (p5) -- (p7);
    \draw [edge] (p6) -- (p7);

\end{tikzpicture}         

            \caption{eine andere}
        \end{figure}
    \end{columns}
\end{frame}
%----------------------------------------------------------------------%SLIDE -

\subsection{Netzwerk}
%----------------------------------------------------------------------%SLIDE -
\begin{frame}
    \frametitle{Netzwerk}
    \begin{columns}
        \column{0.5\textwidth}
        \begin{itemize}
            \item \emph{distributed memory} Modell
            \item ungerichteter Graph
            \item RAMs
            \item feste Verbindungen
            \item verschiedene Topologien
            \item \textbf{send} und \textbf{receive}
        \end{itemize}
        \column{0.5\textwidth}
        \begin{figure}
            \centering
            \begin{tikzpicture}
    [
        processor/.style={rectangle, draw, minimum size=5ex},
    ]
    \node (p1) at (1,0) [processor] {$P_1$};
    \node (p2) at (2,0) [processor] {$P_2$};
    \node (p3) at (3,0) [processor] {$P_3$};
    \node (p4) at (4,0) [processor] {$P_4$};
    \node (pp) at (5.5,0) [processor] {$P_p$};
    \draw (p1) -- (p2) -- (p3) -- (p4);
    \draw [dotted] (p4) -- (pp);
    \draw (pp) -- (6.25,0) -- (6.25,1) -- (0.25,1) -- (0.25,0) -- (p1);
\end{tikzpicture}         
                          

            \caption{Netzwerk in Ringform}
        \end{figure}
    \end{columns}
\end{frame}
%----------------------------------------------------------------------%SLIDE -

%----------------------------------------------------------------------%SLIDE -
\begin{frame}
    \frametitle{Netzwerk}
    \framesubtitle{Beispiel: Matrix-Vektor-Multiplikation}
    \begin{columns}[c]
        \begin{column}{0.5\textwidth}
            \begin{equation*}
                A \cdot x =
                \only<1>{
                    \begin{pmat}[{}]
                        a_{11} & a_{12} & a_{13} & a_{14} \cr
                        a_{21} & a_{22} & a_{23} & a_{24} \cr
                        a_{31} & a_{32} & a_{33} & a_{34} \cr
                        a_{41} & a_{42} & a_{43} & a_{44} \cr
                    \end{pmat}
                    \cdot
                    \begin{pmat}[{}]
                        x_1 \cr
                        x_2 \cr
                        x_3 \cr
                        x_4 \cr
                    \end{pmat}
                }
                \only<2->{
                    \begin{pmat}[{|||}]
                        a_{11} & a_{12} & a_{13} & a_{14} \cr
                        a_{21} & a_{22} & a_{23} & a_{24} \cr
                        a_{31} & a_{32} & a_{33} & a_{34} \cr
                        a_{41} & a_{42} & a_{43} & a_{44} \cr
                    \end{pmat}
                    \cdot
                    \begin{pmat}[{}]
                        x_1 \cr\-
                        x_2 \cr\-
                        x_3 \cr\-
                        x_4 \cr
                    \end{pmat}
                }
            \end{equation*}
        \end{column}
        \begin{column}{0.5\textwidth}
            \only<3>{
                \begin{equation}
                    z_i = A_i \cdot x_i =
                    \begin{bmatrix}
                        a_{1i} \\ a_{2i} \\ a_{3i} \\ a_{4i}
                    \end{bmatrix}
                    \cdot
                    \begin{bmatrix} x_i \end{bmatrix}
                \end{equation}
                \begin{equation}
                    A \cdot x = \sum_{i=1}^4 z_i
                \end{equation}
            }
        \end{column}
    \end{columns}
\end{frame}
%----------------------------------------------------------------------%SLIDE -

%----------------------------------------------------------------------%SLIDE -
\begin{frame}
    \begin{algorithm}[H]
        \caption{Asynchronous Matrix Vector Product on a Ring \cite[S.18]{jaja}}
        \begin{algorithmic}[1]
        \Require (1) Prozessor-ID $i$; (2) Anzahl Prozessoren $p$;
        (3) $i$te Submatrix $A_i$
        (4) $i$te Subvektor $x_i$
        \Ensure $P_i$ berechnet $y = A_1x_1 + \cdots + A_ix_i$
        und gibt das Ergebnis nach rechts. Bei Terminierung hat $P_1$ das
        Ergebnis $z = Ax$.
        \begin{multicols}{2}
            \State Berechne $z_i \gets A_ix_i$
            \If {$i = 1$}
                \State $y \gets 0$
            \Else
                \State \textbf{receive}($y$, left)
            \EndIf
            \State $y \gets y + z_i$
            \State \textbf{send}($y$, right)
            \If {$i = 1$}
                \State \textbf{receive}($z$, left)
            \EndIf
        \end{multicols}
        \end{algorithmic}
    \end{algorithm}
\end{frame}
%----------------------------------------------------------------------%SLIDE -

\subsection{Parallel Random Access Machine}
%----------------------------------------------------------------------%SLIDE -
\begin{frame}
    \frametitle{Parallel Random Access Machine}
    \begin{itemize}
        \item \emph{shared memory} Modell
        \item RAMs
        \item gemeinsamer Speicher
        \item \textbf{global read} und \textbf{global write}
    \end{itemize}
\end{frame}
%----------------------------------------------------------------------%SLIDE -

%----------------------------------------------------------------------%SLIDE -
\begin{frame}
    \frametitle{Parallel Random Access Machine}
    \framesubtitle{Zugriffsbeschränkungen}
    EREW, CREW, common CRCW, arbitrary CRCW, priority CRCW
\end{frame}
%----------------------------------------------------------------------%SLIDE -

%----------------------------------------------------------------------%SLIDE -
\begin{frame}
    \frametitle{Parallel Random Access Machine}
    \framesubtitle{Beispiel}
    an algorithm
\end{frame}
%----------------------------------------------------------------------%SLIDE -

\section{Analyse paralleler Algorithmen}

\subsection{Optimal und effizient}


\subsection{Die Klasse NC}


\printbibliography

\end{document}
