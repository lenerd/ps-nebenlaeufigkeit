\section{Sequentielle Rechnermodelle}
Sequentielle Modelle zeichnen sich dadurch aus, dass Instruktionen der Reihe
nach ausgeführt werden.
Die Zeitkomplexität (\emph{time}) entspricht dabei der Operationenkomplexität
(\emph{work}), die erbracht wurde und berechnet sich aus der Anzahl der
ausgeführten Operationen.
Die Platzkomplexität (\emph{space}) ist modellspezifisch.


\subsection{Straight Line Program}
Ein einfaches Modell ist das Straight Line Program (SPL).
Es besitzt eine Menge von Variablen, mit den Teilmengen der Eingabe- und die
Ausgabevariablen.
Das Programm besteht aus einer Abfolge von Operationen.
Mit jedem Ausführungsschritt wird einer Variablen der Wert einer $k$-stelligen
Operation zugewiesen.
Als Operanden werden Variablen mit bereits definierten Werten übergeben.
\emph{space} ist die Anzahl aller genutzten Variablen abzüglich der
Eingabevariablen.
\emph{time} entspricht der Anzahl der Zuweisungen.
Die Basis eines SPL ist die Menge der Belegungen der Variablen.
Ein SPL wird als boolesch bezeichnet, wenn sowohl die Basis als auch alle
Operationen binär sind.
\cite[S. 9]{reif}


\subsection{Random Access Machine}
Die Random Access Machine (RAM) ist ein häufig genutztes Modell.
Sie ist gleich mächtig zur Turingmaschine; sie lassen sich gegenseitig
simulieren.
Die RAM hat jedoch den Vorteil, ein Modell zu sein, das den realen
Von-Neumann-Rechnern ähnlich ist.

Eine RAM hat eine feste Anzahl von Registern und eine unbegrenzte Anzahl von
Speicherzellen, die durch vorzeichenlose Ganzzahlen adressiert werden.
Je nach Verwendungszweck werden Werte aus verschiedenen Mengen, etwa
$\mathbb{N}$, $\mathbb{R}$ oder andere algebraische Strukturen, als
Speicherinhalte zugelassen.
Der Befehlssatz einer RAM enthält Anweisungen, um den Speicher anzusprechen.
Es ist möglich unter angabe der Adresse Werte aus dem Speicher zu laden
(\emph{load}) und in den Speicher zu schreiben (\emph{store}).
Weitere Befehle für bedingte Verzweigungen oder Arithmetik sind üblicherweise
vorhanden.

Wir werden im Folgenden der Einfachheit halber uniforme Komplexitätsmaße
verwenden.
Eine Zeiteinheit entspricht der Ausführung eines Programmschrittes
und eine Platzeinheit einem Register oder einer Speicherzelle.
Die Zeitkomplexität eines Algorithmus auf einer RAM entspricht also der Anzahl
der ausgeführten Operationen, die Platzkomplexität der Anzahl der verwendeten
Speicherzellen.
\cite[S. 9ff.]{reif}
\cite[S. 184ff.]{fgi2}
