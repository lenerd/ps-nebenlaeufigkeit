\section{Analyse paralleler Algorithmen}

\subsection{Die Klasse $NC$}
\begin{define}
    P: \\
    Die Klasse $P$ ist die Menge aller Probleme, die in polynomieller Zeit entschieden werden können.
    \begin{equation}
        \begin{split}
            P = \left\{ L \, | \, \text{Es gibt ein Polynom } p\colon \mathbb{N} \to \mathbb{R} \right.
            \text{ und eine} \\ \left. \text{$p(n)$-zeitbeschränkte $DTM$ $A$ mit } L = L(A) \right\}
        \end{split}
    \end{equation}\cite[S.205]{fgi1}
\end{define}
Nicht alle Probleme in $P$ sind auch parallelisierbar.
Die Klasse $NC$ (Nick's class, nach Nick Pippenger) enthält die effizient parallel
entscheidbaren Probleme enthalten.
\begin{define}
    NC: \\
    Die Komplexitätsklasse NC ist die Menge der Sprachen $L$,
    die mit einer Eingabe der Länge $n$ durch einen Algorithmus
    in $\mathcal{O}\left( \log^k n \right)$ Zeit auf einer PRAM
    mit $\mathcal{O}\left( n^c \right)$ Prozessoren entschieden werden können.
    $k$ und $c$ sind von $n$ unabhängige Konstanten.
\end{define}
Dass $NC \subseteq P$ gilt, kann gezeigt werden, da jeder Algorithmus auf einer PRAM
einfach in einen sequentiellen Algorithmus umgeformt werden kann.
Ähnlich zu der Beziehung zwischen den Klassen $P$ und $NP$, ist nicht bekannt,
ob auch $P \subseteq NC$ und damit $NC = P$ gilt.
Analog zu den $NP$-vollständigen Problemen, existiert die Klasse der $P$-vollständigen Probleme in $P$.
Es wird vermutet, dass die $P$-vollständigen Probleme nicht in $NC$ enthalten sind.
\cite[S.535]{jaja}

\subsection{P-vollständige Probleme}
