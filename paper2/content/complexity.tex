\section{Analyse paralleler Algorithmen}

\subsection{Die Klasse $NC$}
\begin{define}
    P: \\
    Die Klasse $P$ ist die Menge aller Probleme, die sequentiell in 
    polynomieller Zeit $\mathcal{O}(n^k)$ ($k$ konstant) entschieden werden können.
    \begin{equation}
        \begin{split}
            P = \left\{ L \, | \, \text{Es gibt ein Polynom } p\colon \mathbb{N} \to \mathbb{R} \right.
            \text{ und eine} \\ \left. \text{$p(n)$-zeitbeschränkte $DTM$ $A$ mit } L = L(A) \right\}
        \end{split}
    \end{equation}\cite[S.205]{fgi1}\cite[S.44]{greenlaw}
\end{define}
Probleme werden als effizient parallel entscheidbar bezeichnet, wenn das
hinzunehmen von weiteren Prozessoren eine signifikante Verbesserung der
Laufzeit mit sich bringt.
Diese auch als \emph{highly parallel} bezeichneten Probleme werden in der
Komplexitätsklasse $NC$ (Nick's class, nach Nick Pippenger) zusammengefasst.
\begin{define}
    NC: \\
    Die Klasse $NC$ ist die Menge aller Probleme, die parallel in
    polylogarithmischer Zeit $t(n) = \mathcal{O}(\log^k n)$ auf polynomiell
    vielen Prozessoren $p(n) = \mathcal{O}(n^c)$ entschieden werden können.
    \cite[S.44]{greenlaw}
    % die mit einer Eingabe der Länge $n$ durch einen Algorithmus
    % in $\mathcal{O}\left( \log^k n \right)$ Zeit auf einer PRAM
    % mit $\mathcal{O}\left( n^c \right)$ Prozessoren entschieden werden können.
    % $k$ und $c$ sind von $n$ unabhängige Konstanten.
\end{define}

\begin{lemma}
    Ein Problem ist genau dann sequentiell in polynomieller Zeit entscheidbar,
    wenn es parallel in polynomieller Zeit auf polynomiell vielen Prozessoren
    entscheidbar ist.\cite[S.44]{greenlaw}
    \label{seqiffpar}
\end{lemma}
Ein sequentieller Algorithmus kann auf einer PRAM mit einem Prozessor in der
gleichen Zeit laufen.
Eine RAM kann eine PRAM simulieren, indem die ursprünglich parallel
ausgeführten Instruktionen, nacheinander ausgeführt werden.
$NC \subseteq P$ gilt also aufgrund von Lemma \ref{seqiffpar}.

Ähnlich zu der Beziehung zwischen den Klassen $P$ und $NP$, ist nicht bekannt,
ob auch $P \subseteq NC$ und damit $NC = P$ gilt.
Analog zu den $NP$-vollständigen Problemen, existiert die Klasse der $P$-vollständigen Probleme in $P$.
Es wird vermutet, dass die $P$-vollständigen Probleme nicht in $NC$ enthalten sind.
\cite[S.535]{jaja}

\subsection{P-vollständige Probleme}
