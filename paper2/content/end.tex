\section{Ergebnis}
% 2 + 3
Zunächst wurden Modelle verschiedener Abstraktionsebenen eingeführt, an denen
die parallele Verarbeitung von Daten dargestellt werden können.
% 4
Es wurde gezeigt, dass nicht alle Probleme durch zusätzliche Prozessoren
schneller gelöst werden können.
Diese werden in Komplexitätsklassen eingeordnet und den effizient parallel
lösbaren Problemen gegenübergestellt.
% 5
Mit der Technik Divide and Conquer sowie mit dem Pointer Jumping wurden zwei
Hilfmittel vorgestellt, die Anwendung in parallelen Algorithmen finden.
% 6
An dem Sortieralgorithmus Mergesort wurden verschiedene Ansätze demonstriert,
wie Algorithmen parallelisiert werden können.

Es ist nicht einfach effiziente, parallele Algorithmen zu schreiben.
Vielleicht ist der Aufwand bei kleineren Projekten höher als die zu erwartende
Beschleunigung.
Dennoch sollte man heute und in Zukunft einen parallelen Lösungsansatz in
Betracht ziehen, wenn Aufgaben in höheren Größenordnungen bearbeitet werden
müssen.

