\section{Parallele Rechnermodelle}

\subsection{Circuit}

\subsection{Network}


\subsection{PRAM}
Eine Parallel Random Access Machine, im folgenden PRAM, ist ein oft genutztes Maschinenmodell, um parallele Algorithmen zu analysieren.

Die PRAM besteht aus einem globalen Speicher mit $m$ Speicherzellen sowie $p$ Prozessoren.
Jeder Prozessor ist selbst eine RAM mit lokalem Speicher, die jedoch Lese- und Schreibzugriff auf den globalen Speicher hat.
Jedem Prozessor ist zusätzlich seine eigene Identität bekannt,


\paragraph{work}
Die work einer PRAM ist das Produkt aus der Anzahl der verfügbaren Prozessoren und der Laufzeit.

\cite{reif}

\subsubsection{Access Restrictions}
