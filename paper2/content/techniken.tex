\section{Techniken}

\subsection{Divide and Conquer}

\subsection{Pointer Jumping}
Pointer Jumping wird bei Problemen, die auf Graphen basieren angewandt.
Die Kanten eines Graphen werden in Datenstrukturen als Zeiger auf einen anderen
Knoten gespeichert.
Die Technik erlaubt parallele Verarbeitung von Daten, die zum Beispiel in
gerichteten Bäumen gespeichert sind.
Zeigen die Kanten zu dem Vater eines Knoten; von den Blättern in Richtung
Wurzel, spricht man auch von einem \emph{In-Tree}.

Ein einfaches Beispiel ist die Suche nach den Wurzeln in einem Wald.
\begin{problem}
    Sei $F$ ein Wald aus einer Menge an In-Trees.
    Gegeben ist ein Knoten $j$ in einem der Bäume in $F$.
    Der Wald enthält $n$ Knoten und ist angegeben als ein Array $P$ der Länge
    $n$.
    $P(i)$ enthalte den Vater von $i$.
    Ist $w$ eine Wurzel in $F$, so gelte $P(w) = w$.
    Gesucht ist die Wurzel des Baumes, der den Knoten $j$ enthält.
\end{problem}
%
In jedem Schritt wird der Vater eines Knoten auf den Vater des alten
Vaterknoten gesetzt.
Die Distanz zwischen einem bestimmten Knoten $i$ und $P(i)$ verdoppelt sich bei
jeder Anwendung der Technik, es sei denn $P(i)$ zeigt auf eine Wurzel.
%
\begin{algorithm}
    \caption{Pointer Jumping \cite[S.52]{jaja}}
    \label{alg:pointerjumping}
    \begin{algorithmic}[1]
    \Require Ein Wald aus In-Trees jeweils mit einer Schleifer an der Wurzel.
        Jede Kante zwischen den $n$ Knoten ist als Tupel $(i, P(i))$ mit
        $1 \leq i \leq n$ durch ein Array $P$ definiert.
    \Ensure Für jeden Knoten $i$ die Wurzel $S(i)$ des $i$ beinhaltenden
        Baumes in einem Array $S$.
    \ParDo {$1 \leq i \leq n$}
        \State $S(i) \gets P(i)$
        \While {$S(i) \neq S(S(i))$}
            \State $S(i) \gets S(S(i))$
        \EndWhile
    \EndParDo
    \end{algorithmic}
\end{algorithm}
%
\begin{figure}
    \caption{Ausführung von Algorithmus \ref{alg:pointerjumping}}
\end{figure}
%
\cite[S.52ff]{jaja}
