\section{Betrachtung eines parallelen Mergesort}
Das bereits angesprochene Mergesort (Algorithmus~\ref{alg:mergesort}) ist ein
sequentieller Sortieralgorithmus.
Die Subroutine Merge (Algorithmus~\ref{alg:merge}) hat einen linearen
Zeitaufwand von $\Theta(n)$.
Es lässt sich eine Formel für die sequentielle Laufzeit $MS_1(n)$ aufstellen.
\begin{equation}
    MS_1(n) = \begin{cases}
        \Theta(1) & \text{wenn } n = 1 \\
        2MS_1(n/2) + \Theta(n) & \text{wenn } n > 1
    \end{cases}
\end{equation}
Mithilfe des Master-Theorems lässt sich die durchschnittliche Zeitkomplexität
von $\Theta(n \cdot \log n)$ berechnen.
%
\begin{algorithm}
    \caption{{\rmfamily \textsc{Merge}} \cite[S.31]{cormen}}
    \label{alg:merge}
    \begin{algorithmic}[1]
        \Require Ein Array $A$ mit sortierten Subarrays $A[p:q]$ und
            $A[q+1:r]$.
        \Ensure Das Array $A$ sortiert zwischen $p$ und $r$.
        \setlength\multicolsep{0pt}
        \begin{multicols}{2}
            \State $n_1 \gets q - p + 1$
            \State $n_2 \gets r - q$
            \State $L[1:n_1 +1]$, $R[1:n_2 + 1]$
            \For {$i \gets 1$ \textbf{to} $n_1$}
                \State $L[i] \gets A[p + i - 1]$
            \EndFor
            \For {$j \gets 1$ \textbf{to} $n_2$}
                \State $R[j] \gets A[q + j]$
            \EndFor
            \State $L[n_1 + 1] \gets \infty$
            \State $R[n_2 + 1] \gets \infty$
            \State $i \gets 1$
            \State $j \gets 1$
            \For {$k \gets p$ \textbf{to} $r$}
                \If {$L[i] \leq R[j]$}
                    \State $A[k] \gets L[i]$
                    \State $i \gets i + 1$
                \Else
                    \State $A[k] \gets R[j]$
                    \State $j \gets i + 1$
                \EndIf
            \EndFor
        \end{multicols}
    \end{algorithmic}
\end{algorithm}

Im Folgenden werden verschiedene Ansätze besprochen, um die Ausführungszeit
durch Parallelisierung zu beschleunigen.

\subsection{Ein naiver Ansatz}
Wie in Abschnitt \ref{divide} angedeutet ist der wohl naheliegendste Ansatz, die
beiden rekursiven Aufrufe von Mergesort in den Zeilen 3 und 4 parallel
auszuführen.
Da jeweils ein anderer Abschnitt des Arrays bearbeitet wird, ist die auch ohne
weitere Vorkehrungen möglich (Algorithmus~\ref{alg:mergesort-a}).
\begin{algorithm}
    \caption{{\rmfamily \textsc{Mergesort'}}}
    \label{alg:mergesort-a}
    \begin{algorithmic}[1]
        \Require Ein Array $A$ mit den Indices $p$ und $r$.
        \Ensure Das Array $A$ sortiert zwischen $p$ und $r$.
        \If {$p < r$}
            \State $q \gets \lfloor (p + r) / 2 \rfloor$
            \ParDo {$a,b$ \textbf{in} $[(p,q),(q+1,r)]$}
                \State {\rmfamily \textsc{Mergesort}}$(A, a, b)$
            \EndParDo
            \State {\rmfamily \textsc{Merge}}$(A, p, q, r)$
        \EndIf
    \end{algorithmic}
\end{algorithm}
Wenn eine ausreichende Menge an Prozessoren zur Verfügung steht, ist
$MS'_\infty(n)$ die neue Zeitkomplexität.
\begin{equation}
    MS'_\infty(n) = \begin{cases}
        \Theta(1) & \text{wenn } n = 1 \\
        MS'_\infty(n/2) + \Theta(n) & \text{wenn } n > 1
    \end{cases}
\end{equation}
Es folgt eine durchschnittliche Zeitkomplexität von $\Theta(n)$.
Die durch die Änderung entstandene Beschleunigung entspricht
$\frac{MS_1(n)}{MS'_\infty(n)} = \Theta(\log n)$.
Eine logarithmische Beschleunigung mag sich bei einigen wenigen zusätzlichen
Prozessoren noch lohnen, während es in größeren Dimensionen keinen signifikanten
Unterschied in Bezug auf die Ausführungszeit macht.
\cite[S.797f.]{cormen}

\subsection{Multithreaded Merge}
Der Flaschenhals liegt in der Merge Subroutine.
Die sequentielle Zeitkomplexität ist groß genug um einen wesentlichen Teil der
gesamten Komplexität von Mergesort zu stellen.

Eine Möglichkeit das Mergen zu parallelisieren ist der
Algorithmus~\ref{alg:p-merge} aus \cite{cormen}.

Zunächst wird das mittlere Element des ersten Arrays gewählt.
Es befindet sich an der Stelle $q_1$ und besitzt den Wert $x$.
Da vorausgesetzt wird, dass es sich um ein sortiertes Array handelt, besitzen
alle Elemente vor der Position $q_1$ einen Wert $\leq x$ und alle dahinter
Wert $\geq x$.
Mithilfe einer Binären Suche wird die Position $q_2$ in dem zweiten Array
gefunden, an der man das gewählte Element einsortieren müsste, damit alle
Elemente vor $q_2$ einen Wert $< x$ besitzen.
Nun wurden beide Arrays (durch $q_1$ und $q_2$ getrennt) in jeweils zwei
Teilarrays aufgeteilt.
Auf diese wird parallel, rekursiv der \textsc{P-Merge} Algorithmus angewandt.
Eine schematische Darstellung ist der Abbildung~\ref{fig:p-merge} zu entnehmen.

\begin{figure}
    \centering
    % multithreaded merge schema
\begin{tikzpicture}
    [
        node distance=0cm,
        scale=1,
    ]
    \tikzstyle{subarray}=[
        draw,
        rectangle,
        minimum height=1.6em,
    ]
    \tikzstyle{arrow}=[
        -latex,
        line width=1pt,
    ]
    \tikzstyle{low}=[
        fill=lightgray,
    ]
    \tikzstyle{high}=[
        fill=gray,
    ]

    \node (T) at (-1,2) [] {$T$};
    \node (Tldots) [subarray, right=0cm of T] {$\dots$};

    \node (Tll)    [subarray, low, right=0cm of Tldots, minimum width=2cm] {$\leq x$};
    \node (Tx)     [subarray, right=0cm of Tll, minimum width=0.5cm] {$x$};
    \node (Tlh)    [subarray, high, right=0cm of Tx, minimum width=2cm] {$\geq x$};
    \node (Tmdots) [subarray, right=0cm of Tlh] {$\dots$};
    \node (Thl)    [subarray, low, right=0cm of Tmdots, minimum width=2cm] {$< x$};
    \node (Thh)    [subarray, high, right=0cm of Thl, minimum width=2cm] {$\geq x$};
    \node (Thdots) [subarray, right=0cm of Thh] {$\dots$};

    \node (p1) at (Tldots.north east) [yshift=1em] {$p_1$};
    \node (r1) at (Tmdots.north west) [yshift=1em] {$r_1$};
    \node (p2) at (Tmdots.north east) [yshift=1em] {$p_2$};
    \node (r2) at (Thdots.north west) [yshift=1em] {$r_2$};

    \node (q1) at (Tx.north)       [yshift=1em] {$q_1$};
    \node (q2) at (Thl.north east) [yshift=1em] {$q_2$};

    \node (A) at (0,0) [] {$A$};
    \node (Aldots) [subarray, right=0cm of A] {$\dots$};
    \node (Al)     [subarray, low, right=0cm of Aldots, minimum width=3cm] {$\leq x$};
    \node (Ax)     [subarray, right=0cm of Al, minimum width=0.5cm] {$x$};
    \node (Ah)     [subarray, high, right=0cm of Ax, minimum width=3cm] {$\geq x$};
    \node (Ahdots) [subarray, right=0cm of Ah] {$\dots$};

    \node (q3) at (Ax.south)      [yshift=-1em] {$q_3$};
    \node (p3) at (Al.south west) [yshift=-1em] {$p_3$};
    \node (r3) at (Ah.south east) [yshift=-1em] {$r_3$};

    \draw [arrow, dashed] (Tx.south) -- (Ax.north);
    \draw [arrow, gray] (Tll.south) -- (Al.north);
    \draw [arrow, gray] (Thl.south) -- (Al.north);
    \draw [arrow, darkgray] (Tlh.south) -- (Ah.north);
    \draw [arrow, darkgray] (Thh.south) -- (Ah.north);

\end{tikzpicture}         

    \caption{Multithreaded Merging \cite[S.798]{cormen}}
    \label{fig:p-merge}
\end{figure}

\begin{algorithm}
    \caption{\textsc{P-Merge} \cite[S.800]{cormen}}
    \label{alg:p-merge}
    \begin{algorithmic}[1]
        \Require Zwei Subarrays $T[p_1:r_1]$ und $T[p_2:r_2]$;
            Ziel $A[p_3]$
        \Ensure Die Subarrays von $T$ vereinigt an Position $p_3$ in $A$
        \State $n_1 \gets r_1 - p_1 + 1$
        \State $n_2 \gets r_2 - p_2 + 1$
        \If {$n_1 < n_2$}
            \State exchange $p_1$ with $p_2$
            \State exchange $r_1$ with $r_2$
            \State exchange $n_1$ with $n_2$
        \EndIf
        \If {$n_1 = 0$}
            \State \textbf{return}
        \Else
            \State $q_1 \gets \lfloor ( p_1 + r_1 ) / 2 \rfloor$
            \State $q_2 \gets \textsc{Binary-Search}(T[q_1], T, p_2, r_2)$
            \State $q_3 \gets p_3 + (q_1 - p_1) + (q_2 - p_2)$
            \State $A[q_3] \gets T[q_1]$
            \ParDo {$a,b,c,d,e$ \textbf{in} $[(p_1, q_1 - 1, p_2, q_2 - 1, p_3),
                    (q_1 + 1, r_1, q_2, r_2, q_3 + 1)]$}
                \State $\textsc{P-Merge}(T,a, b, c, d, A, e)$
            \EndParDo
        \EndIf
    \end{algorithmic}
\end{algorithm}

Die Zeitkomplexität des parallelen Merge berechnet mit dem Master-Theorem sich
wie folgt:
\begin{equation}
    \begin{gathered}
        PM_\infty(n) =
        \underbrace{PM_\infty\left( \frac{3}{4}n \right)}_{\text{rekursives Merge}}
        + \underbrace{\Theta(\log n)}_{\text{binäre Suche}}
        \\
        \Rightarrow PM_\infty(n) = \Theta(\log^2 n)
    \end{gathered}
\end{equation}
Die Beschleunigung im Gegensatz zu dem sequentiellen Merge
(Algorithmus~\ref{alg:merge}) beträgt $\Theta(\frac{n}{\log^2 n})$
Die Beschleunigung gilt ebenfalls für das kompletten parallelen Mergesort
(Algorithmus~TODO).
Die Laufzeit ist die folgende:
\begin{equation}
    \begin{gathered}
        \begin{split}
            PMS_\infty (n) &= PMS_\infty(n/2) + PM_\infty(n) \\
            &= PMS_\infty(n/2) + \Theta(\log^2 n)
        \end{split} \\
        \Rightarrow PMS_\infty(n) = \Theta(\log^3 n)
    \end{gathered}
\end{equation}
Eine polylogarithmische Zeitkomplexität ist eine deutliche Verbesserung im
Gegensatz zu dem ersten Versuch, Mergesort zu parallelisieren.
Weitergehende Überlegungen führen zu schnelleren Algorithmen, welche in
$\mathcal{O}(\log n)$ laufen.
Für eine ausführliche Beschreibung wird auf \cite[Kapitel 10]{reif} und
\cite[S.163]{jaja} verwiesen.
