\subsection{Das Schedulingprinzip}
Gegeben sei ein Problem welches durch einen parallelen Algorithmus gelöst wird.
Sei $W_i(n)$ die Anzahl der Operationen zu Zeit $i$ mit $1 \leq i \leq T(n)$.
Eine Maschine mit $p$ Prozessoren simuliert nun die Menge von $W_i(n)$
Operationen, die parallel im Schritt $i$ ausgeführt werden.
Die Simulation läuft in einer Zeit von $< \lceil \frac{W_i(n)}{p} \rceil$
parallel auf $p$ Prozessoren für jeden Zeitslot $i$.
Ist die Simulation erfolgreich, hat der Algorithmus auf einer PRAM mit $p$
Prozessoren hat der Algorithmus eine Laufzeit von
$T_p(n) = \mathcal{O} \left( \frac{W(n)}{p} + T(n) \right)$
\cite[S.28]{jaja}

Während work die Anzahl ausgeführter Operationen bezeichnet, ist \emph{cost}
abhängig von der Menge an Prozessoren, die zur Verfügung steht
$C_p = T_p(n) \cdot p = \mathcal{O} (W(n) + T(n) \cdot p)$.
Werden Prozessoren in Berechnungsschritten nicht genutzt so werden die freien
Resourcen von work nicht berücksichtigt, con cost jedoch schon.
Daher gilt immer $W(n) \leq C_p(n)$.
