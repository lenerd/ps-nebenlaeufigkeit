\documentclass[a4paper]{llncs}

\usepackage[utf8]{inputenc}
\usepackage[T1]{fontenc}
\usepackage{lmodern}
\usepackage[ngerman]{babel}

\setcounter{tocdepth}{3}
\makeatletter
\renewcommand*\l@author[2]{}
\renewcommand*\l@title[2]{}
\makeatother

\usepackage{amsmath, amsfonts}
\spnewtheorem{define}{Definition}{\bfseries}{\rmfamily}
\usepackage{tikz}
\usetikzlibrary[arrows]

\usepackage{algorithm}
\usepackage{algpseudocode}
\renewcommand{\algorithmicrequire}{\textbf{Input:}}
\renewcommand{\algorithmicensure}{\textbf{Output:}}
\floatname{algorithm}{Algorithmus}
% declaration of the new block
\algblock{ParDo}{EndParDo}
% customising the new block
\algnewcommand\algorithmicparfor{\textbf{for}}
\algnewcommand\algorithmicpardo{\textbf{pardo}}
\algnewcommand\algorithmicendparfor{\textbf{end\ pardo}}
\algrenewtext{ParDo}[1]{\algorithmicparfor\ #1\ \algorithmicpardo}
\algrenewtext{EndParDo}{\algorithmicendparfor}

\usepackage[style=alphabetic, backend=biber]{biblatex}
\bibliography{../parallele-algorithmen} 

\begin{document}

\pagestyle{headings}

\title{Parallele Algorithmen}

\author{Lennart Braun}

\institute{
    Universität Hamburg,
    Fakultät für Mathematik, Informatik und Naturwissenschaften,
    Fachbereich Informatik,
    Arbeitsbereich TGI,
    Proseminar Nebenläufigkeit SS 14
    \email{3braun@informatik.uni-hamburg.de}
}

\maketitle

\begin{abstract}
Die Ausarbeitung gibt einen Überblick über parallele Algorithmen.
Dabei wird auf verschiedene Rechnermodelle sowie auf die Komplexität
eingegangen.
Es werden grundlegende Techniken betrachtet, mit denen parallele Algorithmen
entwickelt werden können.

\keywords{
    Algorithmen,
    PRAM,
    Modelle,
    Komplexität,
    NC,
    P-vollständig
}
\end{abstract}

\tableofcontents

\newpage

\section{Einleitung}

\subsection{Motivation}
Computer werden immer leistungsfähiger.
Bis zur Mitte des ersten Jahrzehnts nach der Jahrtausendwende stieg der
Prozessortakt kontinuierlich an und die Ausführung einzelner Instruktionen
wurde optimiert.
Allerdings stagniert der Trend seitdem.
Es sind physikalische Limits erreicht; ein weiterer Anstieg kann nur noch
mit hohem Energieverbrauch und viel Abwärme erkauft werden.
Der Nutzen wird einschränkt.
Die Aussagen bedeuten jedoch nicht, dass moderne Prozessoren nicht effizienter
sein können.
Die Multicore-Architekturen setzen mehrere Prozessoren auf einem Chip ein,
zudem wird Hyper-Threading standardmäßig genutzt.
Aber ein Programm läuft nicht von allein viermal so schnell, wenn man es auf
einer Maschine mit Dualcore Prozessor einsetzt.
Im Gegensatz zur Erhöhung der Taktfrequenz, kann Software nicht einfach mehrere
Kerne verwenden.
Wie Herb Sutter in seinem Artikel „The Free Lunch Is Over“
\cite{sutterlunch} beschreibt, ist es heute notwendig, die Programme anzupassen
und es auszunutzen, dass nebenläufige Ausführung möglich ist.
Man kann nicht mehr darauf vertrauen, dass die Software mit jeder Generation
von Prozessoren schneller läuft.
Die Algorithmen müssen angepasst und parallelisiert werden, falls möglich, um
eine effiziente Auslastung von modernen Computer zu gewährleisten.

Diese Ausarbeitung soll einen groben Überblick über parallele Algorithmen geben.
Es werden theoretische Modelle vorgestellt, anhand derer Algorithmen analysiert
werden können, und es wird auf die Frage eingegangen, welche Probleme überhaupt
parallel gelöst werden können.

\subsection{Der Begriff Parallelität}

\subsection{Aufbau}
In den ersten beiden Abschnitten werden abstrakte Modelle vorgestellt, anhand
derer sequentielle und parallele Algorithmen betrachtet werden können.
Kapitel 4 behandelt die Komplexitätsklassen der beobachteten Algorithmen.
Es wird darauf eingegangen, welche Probleme sich überhaupt effizient parallel
gelöst werden können.

Im Laufe des Textes werden verschiedene Angaben für Komplexität verwendet.
Dabei wird ein uniformes Komplexitätsmaß verwendet.
Die Menge an verwendeten Variablen, in denen Werte während der Ausführung
zwischengespeichert werden, ist die Platzkomplexität ($space$).
Die Zeitkomplexität ($time$) eines Algorithmus gibt die Anzahl der
Zeiteinheiten an, die zur Ausführung gebraucht werden.
Eine Anweisung benötigt eine Zeiteinheit.
Allerdings können insbesondere bei parallelen Algorithmen Anweisungen parallel
ausgeführt werden.
Die Zeitkomplexität entspricht daher nicht unbedingt der Anzahl ausgeführter
Anweisungen ($work$).
Diese wird auch als Operationenkomplexität bezeichnet.
Des weiteren wird bei parallelen Algorithmen die Anzahl der verwendeten
Prozessoren betrachtet.

\section{Sequentielle Rechnermodelle}
Sequentielle Modelle zeichnen sich dadurch aus, dass Instruktionen der Reihe
nach ausgeführt werden.
Die Zeitkomplexität (\emph{time}) entspricht dabei der Operationenkomplexität
(\emph{work}), die erbracht wurde und berechnet sich aus der Anzahl der
ausgeführten Operationen.
Die Platzkomplexität (\emph{space}) ist modellspezifisch.


\subsection{Straight Line Program}
Ein einfaches Modell ist das Straight Line Program (SPL).
Es besitzt eine Menge von Variablen, mit den Teilmengen der Eingabe- und die
Ausgabevariablen.
Das Programm besteht aus einer Abfolge von Operationen.
Mit jedem Ausführungsschritt wird einer Variablen der Wert einer $k$-stelligen
Operation zugewiesen.
Als Operanden werden Variablen mit bereits definierten Werten übergeben.
\emph{space} ist die Anzahl aller genutzten Variablen abzüglich der
Eingabevariablen.
\emph{time} entspricht der Anzahl der Zuweisungen.
Die Basis eines SPL ist die Menge der Belegungen der Variablen.
Ein SPL wird als boolesch bezeichnet, wenn sowohl die Basis als auch alle
Operationen binär sind.
\cite[S. 9]{reif}


\subsection{Random Access Machine}
Die Random Access Machine (RAM) ist ein häufig genutztes Modell.
Sie ist gleich mächtig zur Turingmaschine; sie lassen sich gegenseitig
simulieren.
Die RAM hat jedoch den Vorteil, ein Modell zu sein, das den realen
Von-Neumann-Rechnern ähnlich ist.

Eine RAM hat eine feste Anzahl von Registern und eine unbegrenzte Anzahl von
Speicherzellen, die durch vorzeichenlose Ganzzahlen adressiert werden.
Je nach Verwendungszweck werden Werte aus verschiedenen Mengen, etwa
$\mathbb{N}$, $\mathbb{R}$ oder andere algebraische Strukturen, als
Speicherinhalte zugelassen.
Der Befehlssatz einer RAM enthält Anweisungen, um den Speicher anzusprechen.
Es ist möglich unter angabe der Adresse Werte aus dem Speicher zu laden
(\emph{load}) und in den Speicher zu schreiben (\emph{store}).
Weitere Befehle für bedingte Verzweigungen oder Arithmetik sind üblicherweise
vorhanden.

Wir werden im Folgenden der Einfachheit halber uniforme Komplexitätsmaße
verwenden.
Eine Zeiteinheit entspricht der Ausführung eines Programmschrittes
und eine Platzeinheit einem Register oder einer Speicherzelle.
Die Zeitkomplexität eines Algorithmus auf einer RAM entspricht also der Anzahl
der ausgeführten Operationen, die Platzkomplexität der Anzahl der verwendeten
Speicherzellen.
\cite[S. 9ff.]{reif}
\cite[S. 184ff.]{fgi2}

\section{Parallele Rechnermodelle}
Eine paralleles Modell zeichnet sich dadurch aus, dass mehrere Instruktionen gleichzeitig von Prozessoren ausgeführt werden.
Eine parallele Maschine arbeitet \emph{synchron}, wenn in jedem Schritt alle Prozessoren gleichzeitig eine Instruktion ausführen.
Die Instruktionen müssen nicht unbedingt identisch sein.
Andernfalls arbeitet die Maschine \emph{asynchron}.
Führen alle Prozessoren die gleichartige Instruktionen zur gleichen Zeit aus, wird dies \emph{Single Instruction Multiple Data (SIMD)} genannt; unterscheiden sich die Programme, so lautet die Bezeichnung \emph{Multiple Instructions Multiple Data (MIMD)}.

\subsection{Circuit}
Ein Circuit ist ein Repräsentation als gerichteter azyklischer Graph (engl. directed acyclic graph, DAG) mit benannten Knoten.
Knoten ohne eingehende Kanten, die Blätter, stellen die Eingaben dar, Knoten ohne ausgehende Kanten, die Wurzeln, die Ausgaben.
Die Knoten mit eingehenden Kanten sind mit den Operationen beschriftet, die jeweils in einer Zeiteinheit über maximal zwei Operanden, den eingehenden Kanten, ausgeführt werden.
Die Zeitkomplexität eines Circuit ist die Tiefe des DAG.

Ein Circuit kann aus einem Straight Line Program abgeleitet werden, indem jede Variable, der mehrfach ein Wert zugewiesen wird, durch entsprechend viele neue Variablen ersetzt wird.
Die Knoten des Circuits entsprechen dann den verschiedenen Variablen aus dem SLP.
Die Blätter entsprechen den Eingabevariablen.
Die restlichen Knoten sind mit den Operationen der entsprechenden Variablenzuweisung beschriftet und haben eintreffende Kanten ausgehend von den Abhängigkeiten der Zuweisung.
Ist das SPL boolesch, so spricht man von einem \emph{boolean circuit}.
\cite[S.7]{jaja}\cite[S.11f.]{reif}

\paragraph{Beispiel}
Zwei verschiedene Ansätze, die Summe aus acht Zahlen zu bilden, sind in
Abbildung \ref{circuit} dargestellt.
Circuit $A$ berechnet alle sieben Präfixsummen, addiert alle acht Zahlen
nacheinander.
$A$ besitzt eine Zeitkomplexität von $time = 7$.
$B$ hingegen berechnet Teilsummen parallel.
Der Baum ist flacher, der Algorithmus hat eine Zeitkomplexität von $time = 3$.

\begin{figure}
    \centering
    % two circuits computing the sum of eight numbers
\begin{tikzpicture}
    [
        vertex/.style={circle, draw, scale=1, minimum size=3ex,},
        edge/.style={-latex,},
    ]

    % parallel dag
    \node (a1) at (0,0)     [vertex] {$A_1$};
    \node (a2) at (1,0)     [vertex] {$A_2$};
    \node (a3) at (2,0)     [vertex] {$A_3$};
    \node (a4) at (3,0)     [vertex] {$A_4$};
    \node (a5) at (4,0)     [vertex] {$A_5$};
    \node (a6) at (5,0)     [vertex] {$A_6$};
    \node (a7) at (6,0)     [vertex] {$A_7$};
    \node (a8) at (7,0)     [vertex] {$A_8$};
    \node (p1) at (0.5,1.5) [vertex] {$+$};
    \node (p2) at (2.5,1.5) [vertex] {$+$};
    \node (p3) at (4.5,1.5) [vertex] {$+$};
    \node (p4) at (6.5,1.5) [vertex] {$+$};
    \node (p5) at (1.5,3)   [vertex] {$+$};
    \node (p6) at (5.5,3)   [vertex] {$+$};
    \node (p7) at (3.5,4.5) [vertex] {$+$};
    \draw [edge] (a1) -- (p1);
    \draw [edge] (a2) -- (p1);
    \draw [edge] (a3) -- (p2);
    \draw [edge] (a4) -- (p2);
    \draw [edge] (a5) -- (p3);
    \draw [edge] (a6) -- (p3);
    \draw [edge] (a7) -- (p4);
    \draw [edge] (a8) -- (p4);
    \draw [edge] (p1) -- (p5);
    \draw [edge] (p2) -- (p5);
    \draw [edge] (p3) -- (p6);
    \draw [edge] (p4) -- (p6);
    \draw [edge] (p5) -- (p7);
    \draw [edge] (p6) -- (p7);

    \node at (5,5) {$B$};

    % sequential dag
    \node (b1) at (-3,0)    [vertex] {$A_1$};
    \node (b2) at (-2,0)    [vertex] {$A_2$};

    \node (q1) at (-2.5,1)  [vertex] {$+$};
    \node (b3) at (-1.5,1)  [vertex] {$A_3$};

    \node (q2) at (-2,2)    [vertex] {$+$};
    \node (b4) at (-1,2)    [vertex] {$A_4$};

    \node (q3) at (-1.5,3)  [vertex] {$+$};
    \node (b5) at (-0.5,3)  [vertex] {$A_5$};

    \node (q4) at (-1,4)    [vertex] {$+$};
    \node (b6) at (0,4)     [vertex] {$A_6$};

    \node (q5) at (-0.5,5)  [vertex] {$+$};
    \node (b7) at (0.5,5)   [vertex] {$A_7$};

    \node (q6) at (0,6)     [vertex] {$+$};
    \node (b8) at (1,6)     [vertex] {$A_8$};

    \node (q7) at (0.5,7)   [vertex] {$+$};

    \draw [edge] (b1) -- (q1);
    \draw [edge] (b2) -- (q1);
    \draw [edge] (b3) -- (q2);
    \draw [edge] (b4) -- (q3);
    \draw [edge] (b5) -- (q4);
    \draw [edge] (b6) -- (q5);
    \draw [edge] (b7) -- (q6);
    \draw [edge] (b8) -- (q7);
    \draw [edge] (q1) -- (q2);
    \draw [edge] (q2) -- (q3);
    \draw [edge] (q3) -- (q4);
    \draw [edge] (q4) -- (q5);
    \draw [edge] (q5) -- (q6);
    \draw [edge] (q6) -- (q7);

    \node at (-2,5) {$A$};
\end{tikzpicture}         

    \caption{
        Zwei Circuits: Beide berechnen die Summe aus den Zahlen $A_1$ bis $A_8$.
        \cite[S.8]{jaja}
    }
    \label{circuit}
\end{figure}

\subsection{Network}
Ein Netzwerk aus Prozessoren ist ein ungerichteter Graph $G = (V,E)$.
Jeder Knoten $p \in V$ ist ein Prozessor mit lokalem Speicher. Ein gemeinsamer globaler Speicher ist nicht vorhanden.
Jede Kante $(p_i, p_j) \in E$ stellt eine beidseitige Kommunikationsverbindung zwischen den Prozessoren $p_i$ und $p_j$ dar.
Der Befehlssatz enthält Anweisungen für die Netzwerkkommunikation.
\emph{send(x, i)} schickt die Kopie einer lokalen Variable $x$ an den Prozessor $p_i$.
Anschließend wird die Ausführung des Programmes fortgesetzt.
\emph{receive(y, j)} wartet auf eine Nachricht vom Prozessor $p_j$ und speichert deren Inhalt in der lokalen Variablen $y$.
Im Gegensatz zur Anweisung send, wird die Ausführung so lange unterbrochen, bis tatsächlich eine Nachricht erhalten wird.
Es ist möglich Routingprotokolle zu implementieren, die eine Kommunikation von nicht adjazenten Prozessoren erlaubt.

Die Topologie des Graphen ist Bestandteil des Netzwerkmodells.
Zu betrachtende Eigenschaften des Graphen können Durchmesser, Maximalgrad und Zusammenhang sein.
Einfache Netzwerke sind etwas als lineare Graphen oder Kreisgraphen realisiert.
In Gittergraphen oder Hyperwürfel verfügen die einzelnen Prozessoren über eine höhere Anzahl an Kommunikationsverbindungen.
\cite[S.16f.]{jaja}

\paragraph{Beispiel}
Berechnung des Produktes $y = Ax$ aus einer $n \times n$-Matrix $A$ und einem
Vektor $x$ mit der Länge $n$ auf einem Ring aus $p$ Prozessoren (Abb. \ref{ring}).
$n$ sei ein vielfaches von $p$ mit $r = \frac{n}{p}$.
Die Matrix $A$ wird in $p$ Spalten $A_i$ der Form $n \times r$,
der Vektor $x$ in $p$ Abschnitte der Länge $r$ aufgeteilt.
Die Prozessoren berechnen das Produkt $z_i = A_ix_i$.
Die Summe aller $z_i$ ist der Ergebnisvektor $y = \sum_{i=1}^p z_i$.

\begin{figure}
    \centering
    % Ring mit p Prozessoren
\begin{tikzpicture}
    [
        processor/.style={rectangle, draw, minimum size=5ex},
        scale=1.75,
    ]
    \node (p1) at (1,0) [processor] {$P_1$};
    \node (p2) at (2,0) [processor] {$P_2$};
    \node (p3) at (3,0) [processor] {$P_3$};
    \node (p4) at (4,0) [processor] {$P_4$};
    \node (pp) at (5.5,0) [processor] {$P_p$};
    \draw (p1) -- (p2) -- (p3) -- (p4);
    \draw [dotted] (p4) -- (pp);
    \draw (pp) -- (6.25,0) -- (6.25,1) -- (0.25,1) -- (0.25,0) -- (p1);
\end{tikzpicture}         

    \caption{Netzwerk in Form eines Rings}
    \label{ring}
\end{figure}

\begin{algorithm}
    \caption{Asynchronous Matrix Vector Product on a Ring \cite[S.18]{jaja}}
    \begin{algorithmic}[1]
    \Require (1) Nummer des Prozessors $i$; (2) Anzahl der Prozessoren $p$;
    (3) die $i$te Submatrix $B = A(1:n, (i-1)r+1:ir)$ der Größe $n \times r$ mit $r = \frac{n}{p}$;
    (4) der $i$te Subvektor $w = x((i-1)r+1:ir)$ der Länge $r$
    \Ensure Prozessor $P_i$ berechnet den Vektor $y = A_1x_1 + \cdots + A_ix_i$
    und gibt das Ergebnis nach rechts. Bei Terminierung hat $P_1$ das Ergebnis $Ax$.
    \State Berechne das Prudukt $z \gets Bw$
    \If {$i = 1$}
        \State $y \gets 0$
    \Else
        \State \textbf{receive}($y$, left)
    \EndIf
    \State $y \gets y + z$
    \State \textbf{send}($y$, right)
    \If {$i = 1$}
        \State \textbf{receive}($y$, left)
    \EndIf
    \end{algorithmic}
\end{algorithm}


\subsection{Parallel Random Access Machine}
Eine Parallel Random Access Machine, im folgenden PRAM, ist ein oft genutztes Maschinenmodell, um parallele Algorithmen zu analysieren.
Als eine Erweiterung der RAM, besteht sie aus $p$ Prozessoren, von denen jeder wiederum eine RAM ist, und einer globalen Speichereinheit (\emph{shared memory}).
Jeder Prozessor besitzt einen lokalen Speicher und führt ein lokales Programm aus.
Da sich die Instruktionen, die in einer Zeiteinheit ausgeführt werden, grundsätzlich unterscheiden können, gibt es sowohl MIMD als auch SIMD PRAM.
Die RAMs sind nicht direkt miteinander verbunden, verfügen jedoch über die Anweisungen \emph{global read(X,Y)} und \emph{global write(X,Y)}, mit denen Speicherblöcke aus bzw. in dem shared Memory kopiert werden können.
Der Umfang der Kommunikation zwischen den Prozessoren ist die Menge an Daten, welche zwischen den lokalen und dem globalen Speicher übermittelt wird.
Jedem Prozessor ist zusätzlich seine eigene Identität bekannt, so dass bedingte Anweisungen ermöglichen, Blöcke nur auf bestimmten Prozessoren auszuführen.
Die Operationenkomplexität einer PRAM ist das Produkt aus der Anzahl der verwendeten Prozessoren und der Zeitkomplexität.
\cite[S.9ff.]{jaja}


\subsubsection{Zugriffsbeschränkungen}
Für PRAM Modelle existieren verschiedene Beschränkungen in Bezug auf simultanen Zugriff auf den globalen Speicher.
Die verschiedenen Modelle sind unterschiedlich mächtig. Sie lassen sich unter erhöhtem Ressourcenverbrauch jedoch gegenseitig simulieren.
Im Folgenden sind die üblichen Variationen in aufsteigender Mächtigkeit aufgelistet.
Bei der EREW PRAM kann ein Prozessor zur Zeit eine Speicherzelle auslesen (\emph{exclusive read}) und schreiben (\emph{exclusive write}).
Die CREW lässt mehrere Prozessoren gleichzeitig aus der selben Speicherzelle lesen (\emph{concurrent read}), der Schreibzugriff ist jedoch auf einen Prozessor pro Zeiteinheit begrenzt.
Die CRCW PRAM erlaubt sowohl simultanen Lesezugriff als auch simultanen Schreibzugriff (\emph{concurrent write}).
Bei letzterem können Konflikte entstehen, wenn verschiedene Prozessoren versuchen, gleichzeitig in die selbe Speicherzelle zu schreiben. Diese werden durch verschiedene Subtypen behandelt werden.
Die \emph{common CRCW} lässt einen gleichzeitigen Schreibzugriff nur zu, falls alle beteiligten Prozessoren den gleichen Wert schreiben.
Beim \emph{arbitrary CRCW} schreibt ein beliebiger Prozessor in die Speicherzelle.
Die \emph{priority CRCW} gibt dem beteiligten Prozessor mit der höchsten Priorität (z.\,B. mit der niedrigsten ID) den Schreibzugriff.
Bei der Entwicklung von Algorithmen muss auf die Eigenschaften der verwendeten PRAM eingegangen werden. Beispielsweise darf es bei einem arbitrary CRCW keine Rolle spielen, welcher der Prozessoren in einem Schritt den Schreibzugriff bekommt.
\cite[S.14f.]{jaja}

\section{Komplexität}

\subsection{Die Klasse $NC$}
\begin{define}
    Die Klasse $P$ ist die Menge aller Probleme, die sequentiell in 
    polynomieller Zeit $\mathcal{O}(n^k)$ ($k$ konstant) entschieden werden können.
    \begin{equation}
        \begin{split}
            P = \left\{ L \, | \, \text{Es gibt ein Polynom } p\colon \mathbb{N} \to \mathbb{R} \right.
            \text{ und eine} \\ \left. \text{$p(n)$-zeitbeschränkte $DTM$ $A$ mit } L = L(A) \right\}
        \end{split}
    \end{equation}
    $FP$ sei die Menge aller Funktionen, die sequentiell in polynomieller Zeit
    berechnet werden können.
    \cite[S.205]{fgi1}\cite[S.44f]{greenlaw}
\end{define}
%
Probleme werden als effizient parallel entscheidbar bezeichnet, wenn das
hinzunehmen von weiteren Prozessoren eine signifikante Verbesserung der
Laufzeit mit sich bringt.
Diese auch als \emph{highly parallel} bezeichneten Probleme werden in der
Komplexitätsklasse $NC$ (Nick's class, nach Nick Pippenger) zusammengefasst.
%
\begin{define}
    Die Klasse $NC$ ist die Menge aller Probleme, die parallel in
    polylogarithmischer Zeit $t(n) = \mathcal{O}(\log^k n)$ auf polynomiell
    vielen Prozessoren $p(n) = \mathcal{O}(n^c)$ entschieden werden können.
    Analog zu $P$ und $FP$ sei $FNC$ die Menge aller Funktionen, die in
    polylogarithmischer Zeit auf polynomiell vielen Prozessoren berechnet
    werden können.
    \cite[S.44f]{greenlaw}
    % die mit einer Eingabe der Länge $n$ durch einen Algorithmus
    % in $\mathcal{O}\left( \log^k n \right)$ Zeit auf einer PRAM
    % mit $\mathcal{O}\left( n^c \right)$ Prozessoren entschieden werden können.
    % $k$ und $c$ sind von $n$ unabhängige Konstanten.
\end{define}
%
\begin{lemma}
    Ein Problem ist genau dann sequentiell in polynomieller Zeit entscheidbar,
    wenn es parallel in polynomieller Zeit auf polynomiell vielen Prozessoren
    entscheidbar ist.\cite[S.44]{greenlaw}
    \label{seqiffpar}
\end{lemma}
%
Ein sequentieller Algorithmus kann auf einer PRAM mit einem Prozessor in der
gleichen Zeit laufen.
Eine RAM kann eine PRAM simulieren, indem die ursprünglich parallel
ausgeführten Instruktionen, nacheinander ausgeführt werden.
$NC \subseteq P$ gilt also aufgrund von Lemma \ref{seqiffpar}.
%
Liegt ein Problem in $P$, ist also sequentiell in polynomieller Zeit
entscheidbar, und es gibt keine Lösung, die in $NC$ liegt, so wird das
Problem als \emph{inherently sequential} bezeichnet.
Um zu entscheiden, ob $NC \neq P$ gilt, reicht es, für ein Problem aus $P$ zu
zeigen, dass es nicht in $NC$ lösbar ist.
Bisher ist jedoch noch kein Beweis bekannt.
Die Situation ähnelt der Frage, ob $P \neq NP$ gilt.

Um Relationen zwischen Problemen zu beschreiben, wird der Begriff der
Reduktion eingeführt.
Kann man ein Problem $L$ lösen, indem man es in ein anderes Problem $M$ umformt,
für das man einen Algorithmus kennt, so ist $L$ auf $M$ reduzierbar.
%
\begin{define}
    Eine Sprache $L$ heißt many-one-reduzierbar auf eine Sprache $M$
    ($L \leq_m M$), wenn es eine Funktion $f$ gibt, so dass gilt
    \begin{equation}
        x \in L \Leftrightarrow f(x) \in M
    \end{equation}
    $L$ ist $P$-many-one-reduzierbar auf $M$ ($L \leq_m^p M$), wenn $f \in FP$. \\
    $L$ ist $NC$-many-one-reduzierbar auf $M$ ($L \leq_m^{NC} M$), wenn $f \in FNC$.
    \cite[S.47]{greenlaw}
\end{define}
%
Man kann also sagen, wenn $L \leq_m M$ gilt, ist $L$ höchstens so schwer zu
lösen ist wie $M$.
Der durch die Reduktionsfunktion $f$ entstandene Overhead kann je nach Zweck
vernachlässigt werden.
Gelten $f \in FP$ und $M \in P$, so hat der Algorithmus, der $L$ mithilfe von
$M$ löst, eine polynomielle Laufzeit.
Das Entscheiden von $L$ durch Reduktion auf $M$ ist also nicht wesentlich
schwieriger als $M$.
In diesem Fall wurde gezeigt, dass auch $L \in P$ gilt.

\begin{define}
    Ein Problem $L$ in $P$-schwierig bezüglich $NC$, wenn jedes andere Problem
    in $P$ $NC$-many-one-reduzierbar auf $L$ ist. \\
    Ein Problem $L$ ist $P$-vollständig bezüglich $NC$, wenn $L \in P$ und $L$
    $P$-schwierig bezüglich $NC$ ist.
\end{define}
Im folgenden wird vom Bezug auf die Klasse $NC$ ausgegangen, wenn von
$P$-schwierigen oder $P$-vollständigen Problemen gesprochen wird.
%
Nach Definition sind die $P$-vollständigen Probleme, die schwersten in $P$.
Da sich alle diese Probleme aufeinander reduzieren lassen, wäre nur ein highly
parallel Algorithmus für ein einziges Problem dieser Klasse notwendig, um
$NC = P$ zu zeigen.
Bisher ist dies jedoch noch niemandem gelungen.
Man vermutet daher, dass analog zu den $NP$-vollständigen Problemen gilt
\begin{equation}
    \left\{ L \, | \, L \text{ ist $P$-vollständig} \right\} \subseteq P \backslash NC
\end{equation}
Die $P$-vollständigen Probleme sind somit die wahrscheinlichsten Kandidaten
für inherent sequentielle Probleme.
\cite[S.54ff]{greenlaw}\cite[S.535]{jaja}

\subsection{P-vollständige Probleme}

\subsubsection{Bekannte $P$-vollständige Probleme} \hfill \\
In diesem Abschnitt werden Probleme vorgestellt, für die es einen Beweis gibt,
dass sie $P$-vollständig sind.

% TODO: Einleitung
Das übliche Modell für die Berechnung sequentieller Probleme ist die
Turingmaschine.
In der Praxis eingesetzte Systeme ähneln zwar mehr einer RAM, sind jedoch
genauso mächtig.
Die Gemeinsamkeit ist der einzelne Prozessor, der eine Anweisung zur Zeit
verarbeitet bzw. im Fall der Turingmaschine einen Zustandswechsel nach dem
anderen ausführt.
Ist ein Problem inherent sequentiell, so bedeutet es nicht anderes, als dass es
auf einer PRAM oder einem vergleichbaren parallelen Modell nicht wesentlich
schneller lösbar ist als auf einer Turingmaschine.
Eine Turingmaschine zu simulieren ist also ein ideales sequentielles Problem.

% Generic Machine Simulation Problem
\begin{problem}
    Generic Machine Simulation Problem (GMSP) \\
    \textbf{Eingabe}:
    \begin{itemize}
        \item einen String $x$
        \item eine kodierte Beschreibung $\overline{M}$ eine Turingmaschine $M$
        \item einen Integer $t$ in Unärdarstellung
    \end{itemize}
    Der Eingabestring ist $x\#\overline{M}\#1^t$ mit $\#$ als Trennzeichen. \\
    %
    \textbf{Problem}: \\
    Akzeptiert $M$ die Eingabe $x$ in $t$ Schritten?
    \cite[S.215]{greenlaw}
\end{problem}
Dass $GMSP$ in $P$ liegt, ist einfach zu sehen:
Die Simulation kann durch eine universelle Turingmaschine $M_{GMSP}$
durchgeführt werden.
Die Turingmaschine wird Schritt für Schritt mit der Eingabe simuliert, bis sie
entweder akzeptiert oder $t$ Schritte simuliert worden sind.
Im ersten Fall wird akzeptiert, im zweiten wird nicht akzeptiert.
$M_{GMSP}$ entscheidet $GMSP$.
Dass $GMSP$ $P$-schwierig ist kann ebenfalls gezeigt werden.
Sei $L$ eine beliebige Sprache für die gelte $L \in P$.
Dann gibt es eine Turingmaschine $M_L$ mit $L = L(M_L)$.
Man berechne eine obere Grenze $p(n) = \mathcal{O}(n^k)$ ($k$ fest) für die
Zeitkomplexität von $M_L$ auf einer Eingabe der Länge $n$.
$M_{GMSP}$ erhält $M_L$, einen String $x$ und ein das Zeitlimit abhängig von der
Länge des Strings als Eingabe: $x\#\overline{M_L}\#1^{p(|x|)}$.
$L$ enthält $x$ genau dann, wenn $M_{GMSP}$ die Eingabe akzeptiert.
Da $GMSP$ in $P$ liegt und $P$-schwierig ist, ist es auch $P$-vollständig.
\cite[S.57ff]{greenlaw}


% circuit value problem
\begin{problem}
    Circuit Value Problem (CVP) \\
    \textbf{Eingabe}:
    \begin{itemize}
        \item eine kodierte Beschreibung $\overline{\alpha}$ eines
            boolean Circuit $\alpha$
        \item Eingänge $x_1, \ldots, x_n$
        \item einen Ausgang $y$
    \end{itemize}
    \textbf{Problem}: \\
    Ist der Ausgang $y$ wahr unter der Eingabe $x_1, \ldots, x_n$?
\end{problem}

% Game of Life
\begin{problem}
    Game of Life (LIFE) \\
    \textbf{Eingabe}:
    \begin{itemize}
        \item eine Ausgangskonfiguration des Feldes
        \item eine Zeitangabe $t$
        \item eine Zelle $c$ auf dem Spielfeld
    \end{itemize}
    Das Spielfeld ist ein zweidimensionales Gitter und in beiden Dimensionen
    unendlich groß.
    Jede Zelle ist entweder tot oder lebendig und hat acht Nachbarzellen.
    Eine Zelle, die am Zeitpunkt $t$ tot ist, ist bei $t+1$ lebendig,
    wenn sie zur Zeit $t$ genau drei lebende Nachbarzellen hat.
    Eine bei $t$ lebende Zelle bleibt in $t+1$ lebendig, wenn sie bei $t$ zwei
    oder drei lebendige Nachbarzellen hat; andernfalls stirbt sie. \\
    %
    \textbf{Problem}: \\
    Ist die Zelle $c$ nach der Zeit $t$ am Leben?
    \cite[S.211]{greenlaw}
\end{problem}

% Unification
\begin{problem}
    Unifikation (UNIF) \\
    \textbf{Eingabe}:
    \begin{itemize}
        \item zwei Terme $s$ und $t$: Jeder Term besteht aus Variablen und
            Funktionssymbolen.
    \end{itemize}
    Eine Substitution für eine Variable $x$ in einem Term $u$ ersetzt alle
    Vorkommen von $x$ in $u$ durch einen weiteren Term $v$. \\
    %
    \textbf{Problem}: \\
    Existiert eine Folge von Substitutionen $\sigma$, die $s$ und $t$
    unifiziert?
    Zwei Terme sind unifizierbar, wenn es ein $\sigma$ gibt, so dass
    $\sigma(s) = \sigma(t)$ gilt.
    \cite[S.171]{greenlaw}
\end{problem}

\subsubsection{Probleme mit unbekannter Komplexität} \hfill \\
Für folgende Probleme gibt es bisher weder einen Algorithmus in $NC$,
noch einen Beweis, dass sie $P$-vollständig sind.

% Integer GCD
\begin{problem}
    Größter gemeinsamer Teiler (IntegerGCD) \\
    \textbf{Eingabe}:
    \begin{itemize}
        \item zwei positive $n$-bit Integer $a$ und $b$
    \end{itemize}
    \textbf{Problem}: \\
    Berechne den größten gemeinsamen Teiler von $a$ und $b$ ($\gcd(a,b)$).
    \cite[S.229]{greenlaw}
\end{problem}

% Extened GCD
\begin{problem}
    Erweiterter Euklidischer Algorithmus (ExtendedGCD) \\
    \textbf{Eingabe}:
    \begin{itemize}
        \item zwei positive $n$-bit Integer $a$ und $b$
    \end{itemize}
    \textbf{Problem}: \\
    Berechne zwei Integer $s$ und $t$, so dass $as + bt = \gcd(a,b)$ gilt.
    \cite[S.229]{greenlaw}
\end{problem}

% ModPower
\begin{problem}
    Diskrete Exponentialfunktion (ModPower) \\
    \textbf{Eingabe}:
    \begin{itemize}
        \item positive $n$-bit Integer $a$, $b$ und $c$
    \end{itemize}
    \textbf{Problem}: \\
    Berechne $a^b \mod c$.
    \cite[S.231]{greenlaw}
\end{problem}

\section{Techniken}

\subsection{Divide and Conquer}

\subsection{Pointer Jumping}
Pointer Jumping wird bei Problemen, die auf Graphen basieren angewandt.
Die Kanten eines Graphen werden in Datenstrukturen als Zeiger auf einen anderen
Knoten gespeichert.
Die Technik erlaubt parallele Verarbeitung von Daten, die zum Beispiel in
gerichteten Bäumen gespeichert sind.
Zeigen die Kanten zu dem Vater eines Knoten; von den Blättern in Richtung
Wurzel, spricht man auch von einem \emph{In-Tree}.

Ein einfaches Beispiel ist die Suche nach den Wurzeln in einem Wald.
\begin{problem}
    Sei $F$ ein Wald aus einer Menge an In-Trees.
    Gegeben ist ein Knoten $j$ in einem der Bäume in $F$.
    Der Wald enthält $n$ Knoten und ist angegeben als ein Array $P$ der Länge
    $n$.
    $P(i)$ enthalte den Vater von $i$.
    Ist $w$ eine Wurzel in $F$, so gelte $P(w) = w$.
    Gesucht ist die Wurzel des Baumes, der den Knoten $j$ enthält.
\end{problem}
%
In jedem Schritt wird der Vater eines Knoten auf den Vater des alten
Vaterknoten gesetzt.
Die Distanz zwischen einem bestimmten Knoten $i$ und $P(i)$ verdoppelt sich bei
jeder Anwendung der Technik, es sei denn $P(i)$ zeigt auf eine Wurzel.
%
\begin{algorithm}
    \caption{Pointer Jumping \cite[S.52]{jaja}}
    \label{alg:pointerjumping}
    \begin{algorithmic}[1]
    \Require Ein Wald aus In-Trees jeweils mit einer Schleifer an der Wurzel.
        Jede Kante zwischen den $n$ Knoten ist als Tupel $(i, P(i))$ mit
        $1 \leq i \leq n$ durch ein Array $P$ definiert.
    \Ensure Für jeden Knoten $i$ die Wurzel $S(i)$ des $i$ beinhaltenden
        Baumes in einem Array $S$.
    \ParDo {$1 \leq i \leq n$}
        \State $S(i) \gets P(i)$
        \While {$S(i) \neq S(S(i))$}
            \State $S(i) \gets S(S(i))$
        \EndWhile
    \EndParDo
    \end{algorithmic}
\end{algorithm}
%
\begin{figure}
    \caption{Ausführung von Algorithmus \ref{alg:pointerjumping}}
\end{figure}
%
\cite[S.52ff]{jaja}

\section{Betrachtung eines parallelen Mergesort}

\section{Ergebnis}
% 2 + 3
Zunächst wurden Modelle verschiedener Abstraktionsebenen eingeführt, an denen
die parallele Verarbeitung von Daten dargestellt werden können.
% 4
Es wurde gezeigt, dass nicht alle Probleme durch zusätzliche Prozessoren
schneller gelöst werden können.
Diese werden in Komplexitätsklassen eingeordnet und den effizient parallel
lösbaren Problemen gegenübergestellt.
% 5
Mit der Technik Divide and Conquer sowie mit dem Pointer Jumping wurden zwei
Hilfmittel vorgestellt, die Anwendung in parallelen Algorithmen finden.
% 6
An dem Sortieralgorithmus Mergesort wurden verschiedene Ansätze demonstriert,
wie Algorithmen parallelisiert werden können.

Es ist nicht einfach, effiziente, parallele Algorithmen zu schreiben.
Vielleicht ist der Aufwand bei kleineren Projekten höher als die zu erwartende
Beschleunigung.
Dennoch sollte man heute und in Zukunft einen parallelen Lösungsansatz in
Betracht ziehen, wenn Aufgaben in höheren Größenordnungen bearbeitet werden
müssen.



\clearpage
\printbibliography

\end{document}
