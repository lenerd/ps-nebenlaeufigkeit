\subsection{Die Klasse $NC$}
\begin{define}
    Die Klasse $P$ ist die Menge aller Probleme, die sequentiell in 
    polynomieller Zeit $\mathcal{O}(n^k)$ ($k$ konstant) entschieden werden
    können.
    \begin{equation}
        \begin{split}
            P = \left\{ L \, | \, \text{Es gibt ein Polynom }
            p\colon \mathbb{N} \to \mathbb{R} \right.
            \text{ und eine} \\ \left. \text{$p(n)$-zeitbeschränkte $DTM$ $A$
            mit } L = L(A) \right\}
        \end{split}
    \end{equation}
    $FP$ sei die Menge aller Funktionen, die sequentiell in polynomieller Zeit
    berechnet werden können.
    \cite[S.205]{fgi1}\cite[S.44f]{greenlaw}
\end{define}
%
Probleme werden als effizient parallel entscheidbar bezeichnet, wenn das
hinzunehmen von weiteren Prozessoren eine signifikante Verbesserung der
Laufzeit mit sich bringt.
Diese auch als \emph{highly parallel} bezeichneten Probleme werden in der
Komplexitätsklasse $NC$ (Nick's class, nach Nick Pippenger) zusammengefasst.
%
\begin{define}
    Die Klasse $NC$ ist die Menge aller Probleme, die parallel in
    polylogarithmischer Zeit $t(n) = \mathcal{O}(\log^k n)$ auf polynomiell
    vielen Prozessoren $p(n) = \mathcal{O}(n^c)$ entschieden werden können.
    Analog zu $P$ und $FP$ sei $FNC$ die Menge aller Funktionen, die in
    polylogarithmischer Zeit auf polynomiell vielen Prozessoren berechnet
    werden können.
    \cite[S.44f]{greenlaw}
\end{define}
%
\begin{lemma}
    Ein Problem ist genau dann sequentiell in polynomieller Zeit entscheidbar,
    wenn es parallel in polynomieller Zeit auf polynomiell vielen Prozessoren
    entscheidbar ist.\cite[S.44]{greenlaw}
    \label{seqiffpar}
\end{lemma}
%
Ein sequentieller Algorithmus kann auf einer PRAM mit einem Prozessor in der
gleichen Zeit laufen.
Eine RAM kann eine PRAM simulieren, indem die ursprünglich parallel
ausgeführten Instruktionen, nacheinander ausgeführt werden.
$NC \subseteq P$ gilt also aufgrund von Lemma \ref{seqiffpar}.
%
Liegt ein Problem in $P$, ist also sequentiell in polynomieller Zeit
entscheidbar, und es gibt keine Lösung, die in $NC$ liegt, so wird das
Problem als \emph{inherently sequential} bezeichnet.
Um zu entscheiden, ob $NC \neq P$ gilt, reicht es, für ein Problem aus $P$ zu
zeigen, dass es nicht in $NC$ lösbar ist.
Bisher ist jedoch noch kein Beweis bekannt.
Die Situation ähnelt der Frage, ob $P \neq NP$ gilt.

Um Relationen zwischen Problemen zu beschreiben, wird der Begriff der
Reduktion eingeführt.
Kann man ein Problem $L$ lösen, indem man es in ein anderes Problem $M$ umformt,
für das man einen Algorithmus kennt, so ist $L$ auf $M$ reduzierbar.
%
\begin{define}
    Eine Sprache $L$ heißt many-one-reduzierbar auf eine Sprache $M$
    ($L \leq_m M$), wenn es eine Funktion $f$ gibt, so dass gilt
    \begin{equation}
        x \in L \Leftrightarrow f(x) \in M
    \end{equation}
    $L$ ist $P$-many-one-reduzierbar auf $M$ ($L \leq_m^p M$), wenn
    $f \in FP$. \\
    $L$ ist $NC$-many-one-reduzierbar auf $M$ ($L \leq_m^{NC} M$), wenn
    $f \in FNC$.
    \cite[S.47]{greenlaw}
\end{define}
%
Man kann also sagen, wenn $L \leq_m M$ gilt, ist $L$ höchstens so schwer zu
lösen ist wie $M$.
Der durch die Reduktionsfunktion $f$ entstandene Overhead kann je nach Zweck
vernachlässigt werden.
Gelten $f \in FP$ und $M \in P$, so hat der Algorithmus, der $L$ mithilfe von
$M$ löst, eine polynomielle Laufzeit.
Das Entscheiden von $L$ durch Reduktion auf $M$ ist also nicht wesentlich
schwieriger als $M$.
In diesem Fall wurde gezeigt, dass auch $L \in P$ gilt.

\begin{define}
    Ein Problem $L$ in $P$-schwierig bezüglich $NC$, wenn jedes andere Problem
    in $P$ $NC$-many-one-reduzierbar auf $L$ ist. \\
    Ein Problem $L$ ist $P$-vollständig bezüglich $NC$, wenn $L \in P$ und $L$
    $P$-schwierig bezüglich $NC$ ist.
\end{define}
Im folgenden wird vom Bezug auf die Klasse $NC$ ausgegangen, wenn von
$P$-schwierigen oder $P$-vollständigen Problemen gesprochen wird.
%
Nach Definition sind die $P$-vollständigen Probleme, die schwersten in $P$.
Da sich alle diese Probleme aufeinander reduzieren lassen, wäre nur ein highly
parallel Algorithmus für ein einziges Problem dieser Klasse notwendig, um
$NC = P$ zu zeigen.
Bisher ist dies jedoch noch niemandem gelungen.
Man vermutet daher, dass analog zu den $NP$-vollständigen Problemen gilt
\begin{equation}
    \left\{ L \, | \, L \text{ ist $P$-vollständig} \right\}
    \subseteq P \backslash NC
\end{equation}
Die $P$-vollständigen Probleme sind somit die wahrscheinlichsten Kandidaten
für inherent sequentielle Probleme.
\cite[S.54ff]{greenlaw}\cite[S.535]{jaja}
