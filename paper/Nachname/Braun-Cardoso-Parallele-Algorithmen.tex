% !TEX root = ../soziale-netzwerke-gesamtband-version-01.tex
%UTF-8: äöüß
%Nachname-Vorname-Titel2.tex
%
% Erstellerin: Nachname2, Vorname2
% Datum: 06.05.2012
%
%%%%%%%%%%%%%%%%%%%%%%%%%%%%%%%%%%%%%%%%%%%%%%%%%%%%%%%%%%%%%%%%%%%%%%%%%%%%%%%
%
% second contribution with nearly identical text,
% slightly changed contribution head (all entries
% appear as defaults), and modified bibliography
%
% Im zweiten Beitrag habe ich bis auf das markboth nichts geändert.
%
%%%%%%%%%%%%%%%%%%%%%%%%%%%%%%%%%%%%%%%%%%%%%%%%%%%%%%%%%%%%%%%%%%%%%%%%%%%%%%%
\title{Parallele Algorithmen}

\author{Francisco Cardoso \and Lennart Braun}

\institute{Universität Hamburg, Fakultät für Mathematik, Informatik und Naturwissenschaften, Fachbereich Informatik, Arbeitsbereich TGI, Proseminar Nebenläufigkeit SS 14}

\maketitle
\pagebreak
%
% Modify the bibliography environment to call for the author-year
% system. This is done normally with the citeauthoryear option
% for a particular contribution.
%\makeatletter
%\renewenvironment{thebibliography}[1]
%     {\section*{\refname}
%      \small
%      \list{}%
%           {\settowidth\labelwidth{}%
%            \leftmargin\parindent
%            \itemindent=-\parindent
%            \labelsep=\z@
%            \if@openbib
%              \advance\leftmargin\bibindent
%              \itemindent -\bibindent
%              \listparindent \itemindent
%              \parsep \z@
%            \fi
%            \usecounter{enumiv}%
%            \let\p@enumiv\@empty
%            \renewcommand\theenumiv{}}%
%      \if@openbib
%        \renewcommand\newblock{\par}%
%      \else
%        \renewcommand\newblock{\hskip .11em \@plus.33em \@minus.07em}%
%      \fi
%      \sloppy\clubpenalty4000\widowpenalty4000%
%      \sfcode`\.=\@m}
%     {\def\@noitemerr
%       {\@latex@warning{Empty `thebibliography' environment}}%
%      \endlist}
%      \def\@cite#1{#1}%
%      \def\@lbibitem[#1]#2{\item[]\if@filesw
%        {\def\protect##1{\string ##1\space}\immediate
%      \write\@auxout{\string\bibcite{#2}{#1}}}\fi\ignorespaces}
%\makeatother
%Index darf erst hier kommen: Ansonsten falsche Seitenzahlangabe...
\index{personen}{Cardoso, Francisco}
\index{personen}{Braun, Lennart}

%
\markboth{\textit{Nebenläufigkeit SS\/13}: Cardoso, Braun}
{Parallele Algorithmen}
%
%
\begin{abstract}
Wir stellen verschiedene Modelle für parallele Architekturen vor und gehen auf die verschiedenen Eigenschaften ein. Weiterhin wird auf Möglichkeiten eingegangen, Algorithmen zu paralelisieren.
\keywords{parallele Algorithmen, PRAM, Sortieralgorithmen}
\end{abstract}
\pagebreak

\section{Einleitung}
\section{Modelle zu Analyse paralleler Algorithmen}
\subsection{Network Model}
\subsection{PRAM}
\subsubsection{EREW}
\subsubsection{CREW}
\subsubsection{common CRCW}
\subsubsection{arbitrary CRCW}
\subsubsection{priority CRCW}
\section{Parallelrechner}
\subsection{Massiv-parallele Computer}
\subsection{Pipelining}
\section{Parallelisierbarkeit}
\subsection{Embarrassingly parallel problems}
\subsection{Inherently serial problems}
\section{Techniken}
\subsection{Divide and Conquer}
\subsection{Partitionieren}
\subsection{Pointer jumping}
\section{Anwendung paralleler Algorithmen}
\subsection{Allgemein}
\subsection{Sortieralgorithmen}
\subsubsection{Quicksort}
\subsubsection{Mergesort}
\subsubsection{Beispiel Implementationen}

\pagebreak
\nocite{*}
\printbibliography
\pagebreak
\section*{Anforderungen an andere Themen}
Für Implementationen von Algorithmen wäre es gut, wenn die Vorträge zu den Programmiersprachen vor unserem kommen.
% EOF
