% -*- coding: utf-8 -*-
% !TEX root = ../main.tex
%UTF-8: äöüß
%slides.tex

\section{Gliederung}
\label{sec:gliederung}

%----------------------------------------------------------------------%SLIDE -
\begin{frame}
\frametitle{\insertsection} 
    \begin{itemize}
        %\item Abstract
        %\item Einleitung
        \item Modelle zur Analyse paralleler Algorithmen
        \item Voraussetzungen der Parallelisierbarkeit
        \item Scheduling
        \item Parallelisierung am Beispiel von Sortieralgorithmen
        %\item Zusammenfassung
    \end{itemize}
\end{frame}
%----------------------------------------------------------------------%SLIDE -


\section{Literatur}

%----------------------------------------------------------------------%SLIDE -
\begin{frame}
    \frametitle{\insertsection}
    \begin{itemize}
        \item D. Moldt, R. Valk. \emph{Formale Grundlagen der Informatik II: Modellierung \& Analyse.} Wintersemester 2012/2013.
        \item J. J\'aJ\'a. \emph{An Introduction to Parallel Algorithms.} Addison-Wesley Publ. Co., 1992.
        \item V. Kale, E. Solomonik. \emph{Parallel sorting pattern.} 2010.
        \item K. Cleereman. \emph{Speedsort: improving the quicksort algorithm.} 2002.
        \item C. Keßler. \emph{A practical access to the theory of parallel algorithms.} 2004.
    \end{itemize}
\end{frame}
%----------------------------------------------------------------------%SLIDE -


%----------------------------------------------------------------------%SLIDE -
\begin{frame}
    \frametitle{\insertsection}
    \begin{itemize}
        \item J. Reif. \emph{Synthesis of Parallel Algorithms.} Morgan Kaufmann Publishers, 1993.
        \item P. Chaudhuri. \emph{Parallel algorithms.} Prentice Hall, 1992.
        \item S. Rajasekaran, J. Reif. \emph{Handbook of parallel computing: models, algorithms and applications.} Chapman \& Hall/CRC, 2008.
        \item diverse Skripte
    \end{itemize}
\end{frame}
%----------------------------------------------------------------------%SLIDE -


%%% Local Variables: 
%%% mode: latex
%%% TeX-master: "../main"
%%% End: 
