\section{Paralleles Mergesort}

\subsection{Sequentielles Mergesort}
%----------------------------------------------------------------------%SLIDE -
\begin{frame}
    \frametitle{Betrachtung eines parallelen Mergesorts}
    \framesubtitle{Sequential Mergesorts}
    \begin{algorithm}[H]
        \caption{{\rmfamily \textsc{Mergesort}} \cite[S.34]{cormen}}
        \label{alg:mergesort}
        \begin{algorithmic}[1]
            \Require Ein Array $A$ mit den Indices $p$ und $r$.
            \Ensure Das Array $A$ sortiert zwischen $p$ und $r$.
            \If {$p < r$}
                \State $q \gets \lfloor (p + r) / 2 \rfloor$
                \State {\rmfamily \textsc{Mergesort}}$(A, p, q)$
                \State {\rmfamily \textsc{Mergesort}}$(A, q + 1, r)$
                \State {\rmfamily \textsc{Merge}}$(A, p, q, r)$
            \EndIf
        \end{algorithmic}
    \end{algorithm}
\end{frame}
%----------------------------------------------------------------------%SLIDE -

%----------------------------------------------------------------------%SLIDE -
\begin{frame}
    \begin{algorithm}[H]
        \caption{{\rmfamily \textsc{Merge}} \cite[S.31]{cormen}}
        \label{alg:merge}
        \begin{algorithmic}[1]
            \Require Ein Array $A$ mit sortierten Subarrays $A[p:q]$ und
                $A[q+1:r]$.
            \Ensure Das Array $A$ sortiert zwischen $p$ und $r$.
            \setlength\multicolsep{0pt}
            \begin{multicols}{2}
                \State $n_1 \gets q - p + 1$
                \State $n_2 \gets r - q$
                \State $L[1:n_1 +1]$, $R[1:n_2 + 1]$
                \For {$i \gets 1$ \textbf{to} $n_1$}
                    \State $L[i] \gets A[p + i - 1]$
                \EndFor
                \For {$j \gets 1$ \textbf{to} $n_2$}
                    \State $R[j] \gets A[q + j]$
                \EndFor
                \State $L[n_1 + 1] \gets \infty$
                \State $R[n_2 + 1] \gets \infty$
                \State $i \gets 1$
                \State $j \gets 1$
                \For {$k \gets p$ \textbf{to} $r$}
                    \If {$L[i] \leq R[j]$}
                        \State $A[k] \gets L[i]$
                        \State $i \gets i + 1$
                    \Else
                        \State $A[k] \gets R[j]$
                        \State $j \gets i + 1$
                    \EndIf
                \EndFor
            \end{multicols}
        \end{algorithmic}
    \end{algorithm}
\end{frame}
%----------------------------------------------------------------------%SLIDE -

%----------------------------------------------------------------------%SLIDE -
\begin{frame}
    \frametitle{Sequentielle Laufzeit}
    \begin{itemize}
        \item {\rmfamily \textsc{Merge}} läuft in $\Theta(n)$
        \item Laufzeit von {\rmfamily \textsc{Mergesort}}
            \begin{equation}
                MS_1(n) = \begin{cases}
                    \Theta(1) & \text{wenn } n = 1 \\
                    2MS_1(n/2) + \Theta(n) & \text{wenn } n > 1
                \end{cases}
            \end{equation}
        \item {\rmfamily \textsc{Mergesort}} läuft in $\Theta(n \log n)$
    \end{itemize}
\end{frame}
%----------------------------------------------------------------------%SLIDE -

\subsection{1. Ansatz}
%----------------------------------------------------------------------%SLIDE -
\begin{frame}
    \frametitle{Parallelisierung}
    \framesubtitle{1. Ansatz}
    Wieso nicht die beiden rekursiven Aufrufe parallel ausführen?
    \begin{algorithm}[H]
        \caption{{\rmfamily \textsc{Mergesort'}} \cite[S.34]{cormen}}
        \label{alg:mergesort}
        \begin{algorithmic}[1]
            \Require Ein Array $A$ mit den Indices $p$ und $r$.
            \Ensure Das Array $A$ sortiert zwischen $p$ und $r$.
            \If {$p < r$}
                \State $q \gets \lfloor (p + r) / 2 \rfloor$
                %\State $P[1] \gets p - 1$, $P[2] \gets q$, $P[3] \gets r$
                \ParDo {$a,b$ \textbf{in} $[(p,q),(q+1,r)]$}
                    \State {\rmfamily \textsc{Mergesort}}$(A, a, b)$
                \EndParDo
                %\ParDo {$i$ \textbf{in} $1, 2$}
                %    \State {\rmfamily \textsc{Mergesort}}$(A, P[i] + 1, P[i+1])$
                %\EndParDo
                \State {\rmfamily \textsc{Merge}}$(A, p, q, r)$
            \EndIf
        \end{algorithmic}
    \end{algorithm}
\end{frame}
%----------------------------------------------------------------------%SLIDE -

%----------------------------------------------------------------------%SLIDE -
\begin{frame}
    \frametitle{Parallele Laufzeit}
    \framesubtitle{1. Ansatz}
    \begin{itemize}
        \item {\rmfamily \textsc{Merge}} läuft in $\Theta(n)$
        \item Laufzeit von {\rmfamily \textsc{Mergesort'}}
            \begin{equation}
                MS'_\infty(n) = \begin{cases}
                    \Theta(1) & \text{wenn } n = 1 \\
                    MS'_\infty(n/2) + \Theta(n) & \text{wenn } n > 1
                \end{cases}
            \end{equation}
        \item {\rmfamily \textsc{Mergesort'}} läuft in $\Theta(n)$
        \item Speedup
            \begin{equation}
                \frac{MS_1(n)}{MS'_\infty(n)} = \Theta(\log n)
            \end{equation}
    \end{itemize}
\end{frame}
%----------------------------------------------------------------------%SLIDE -

%----------------------------------------------------------------------%SLIDE -
\begin{frame}
    \centering
    \huge
    Wo ist der Flaschenhals? \\

    \pause
    {\rmfamily\textsc{Merge}}
\end{frame}
%----------------------------------------------------------------------%SLIDE -

\subsection{Cormen et al.'s parallel merge}
%----------------------------------------------------------------------%SLIDE -
\begin{frame}
    \frametitle{Cormen et al.'s parallel merge}
    \begin{figure}
        \centering
        % multithreaded merge schema
\begin{tikzpicture}
    [
        node distance=0cm,
        scale=1,
    ]
    \tikzstyle{subarray}=[
        draw,
        rectangle,
        minimum height=1.6em,
    ]
    \tikzstyle{arrow}=[
        -latex,
        line width=1pt,
    ]
    \tikzstyle{low}=[
        fill=lightgray,
    ]
    \tikzstyle{high}=[
        fill=gray,
    ]

    \node (T) at (-1,2) [] {$T$};
    \node (Tldots) [subarray, right=0cm of T] {$\dots$};

    \node (Tll)    [subarray, low, right=0cm of Tldots, minimum width=2cm] {$\leq x$};
    \node (Tx)     [subarray, right=0cm of Tll, minimum width=0.5cm] {$x$};
    \node (Tlh)    [subarray, high, right=0cm of Tx, minimum width=2cm] {$\geq x$};
    \node (Tmdots) [subarray, right=0cm of Tlh] {$\dots$};
    \node (Thl)    [subarray, low, right=0cm of Tmdots, minimum width=2cm] {$< x$};
    \node (Thh)    [subarray, high, right=0cm of Thl, minimum width=2cm] {$\geq x$};
    \node (Thdots) [subarray, right=0cm of Thh] {$\dots$};

    \node (p1) at (Tldots.north east) [yshift=1em] {$p_1$};
    \node (r1) at (Tmdots.north west) [yshift=1em] {$r_1$};
    \node (p2) at (Tmdots.north east) [yshift=1em] {$p_2$};
    \node (r2) at (Thdots.north west) [yshift=1em] {$r_2$};

    \node (q1) at (Tx.north)       [yshift=1em] {$q_1$};
    \node (q2) at (Thl.north east) [yshift=1em] {$q_2$};

    \node (A) at (0,0) [] {$A$};
    \node (Aldots) [subarray, right=0cm of A] {$\dots$};
    \node (Al)     [subarray, low, right=0cm of Aldots, minimum width=3cm] {$\leq x$};
    \node (Ax)     [subarray, right=0cm of Al, minimum width=0.5cm] {$x$};
    \node (Ah)     [subarray, high, right=0cm of Ax, minimum width=3cm] {$\geq x$};
    \node (Ahdots) [subarray, right=0cm of Ah] {$\dots$};

    \node (q3) at (Ax.south)      [yshift=-1em] {$q_3$};
    \node (p3) at (Al.south west) [yshift=-1em] {$p_3$};
    \node (r3) at (Ah.south east) [yshift=-1em] {$r_3$};

    \draw [arrow, dashed] (Tx.south) -- (Ax.north);
    \draw [arrow, gray] (Tll.south) -- (Al.north);
    \draw [arrow, gray] (Thl.south) -- (Al.north);
    \draw [arrow, darkgray] (Tlh.south) -- (Ah.north);
    \draw [arrow, darkgray] (Thh.south) -- (Ah.north);

\end{tikzpicture}         

        \caption{Multithreaded Merging \cite[S.798]{cormen}}
        \label{fig:p-merge}
    \end{figure}
\end{frame}
%----------------------------------------------------------------------%SLIDE -

%----------------------------------------------------------------------%SLIDE -
\begin{frame}
    \frametitle{Cormen et al.'s parallel merge}
    \begin{equation}
        \begin{gathered}
            PM_\infty(n) =
            \underbrace{PM_\infty\left( \frac{3}{4}n \right)}_{\text{rekursives Merge}}
            + \underbrace{\Theta(\log n)}_{\text{Binary Search}}
            \\
            \Rightarrow PM_\infty(n) = \Theta(\log^2 n)
        \end{gathered}
    \end{equation}
    \pause
    Speedup
    \begin{equation}
        \frac{PM_1(n)}{PM_\infty(n)} = \Theta \left( \frac{n}{\log^2 n} \right)
    \end{equation}
\end{frame}
%----------------------------------------------------------------------%SLIDE -

%----------------------------------------------------------------------%SLIDE -
\begin{frame}
    \frametitle{Cormen et al.'s parallel merge}
    \begin{equation}
        \begin{gathered}
            \begin{split}
                PMS_\infty(n) &= PMS_\infty(n/2) + PM_\infty(n) \\
                &= PMS_\infty(n/2) + \Theta(\log^2 n)
            \end{split}
            \\
            \Rightarrow PMS_\infty(n) = \Theta(\log^3 n)
        \end{gathered}
    \end{equation}
    \pause
    Speedup
    \begin{equation}
        \frac{PMS_1(n)}{PMS_\infty(n)} = \Theta \left( \frac{n}{\log^2 n} \right)
    \end{equation}
\end{frame}
%----------------------------------------------------------------------%SLIDE -

\subsection{Es geht schneller!}
%----------------------------------------------------------------------%SLIDE -
\begin{frame}
    \frametitle{Es geht schneller!}
    \begin{itemize}
        \item Joseph J\'aj\'as „Pipelined Merge Sort“ \cite[S.163]{jaja} in
            $\mathcal{O}(n)$
        \item Richard Coles Algorithmen \cite[Kapitel 10]{reif} in
            $\mathcal{O}(\log^2 n)$ und $\mathcal{O}(\log n)$
    \end{itemize}
\end{frame}
%----------------------------------------------------------------------%SLIDE -
