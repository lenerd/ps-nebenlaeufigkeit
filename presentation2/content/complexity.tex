\section{Komplexität}
%----------------------------------------------------------------------%SLIDE -
\begin{frame}
    \frametitle{Komplexität}
    \tableofcontents[
        currentsection,
        hideothersubsections,
        sectionstyle=show/shaded,
    ]
\end{frame}
%----------------------------------------------------------------------%SLIDE -

\subsection{Schedulingprinzip}
%----------------------------------------------------------------------%SLIDE -
\begin{frame}
    \frametitle{Schedulingprinzip}
    Sei $T(n)$ die Laufzeit eines parallelen Algorithmus mit $W_i(n)$
    Operationen zur Zeit $i$.
    \begin{equation}
        \begin{gathered}
            T_p(n) = \sum_{i=1}^{T(n)} \left\lceil \frac{W_i(n)}{P} \right\rceil
            <
            \sum_{i=1}^{T(n)} \left( \frac{W_i(n)}{p} + 1 \right)
            = \frac{W(n)}{p} + T(n) \\
            \Rightarrow T_p(n) = \mathcal{O} \left( \frac{W(n)}{p} + T(n) \right)
        \end{gathered}
    \end{equation}
    Auf $p$ Prozessoren kann der Algorithmus in der Zeit $T_p(n)$ laufen.
\end{frame}
%----------------------------------------------------------------------%SLIDE -

\subsection{Optimalität}
%----------------------------------------------------------------------%SLIDE -
\begin{frame}
    \frametitle{Was ist optimal?}
    Sei $T^\ast(n)$ die bestmögliche sequentielle Laufzeit für ein Problem.
    \begin{definition}[time optimal]
        Ein sequentieller Algorithmus mit $T(n) = \mathcal{O}(T^\ast(n))$.
    \end{definition}
    \begin{definition}[optimal]
        Ein paralleler Algorithmus mit $W(n) = \Theta(T^\ast(n))$.
    \end{definition}
    \begin{definition}[work-time optimal]
        Ein optimaler, paralleler Algorithmus mit der bestmöglichen Laufzeit.
    \end{definition}
\end{frame}
%----------------------------------------------------------------------%SLIDE -

%----------------------------------------------------------------------%SLIDE -
\begin{frame}
    \frametitle{Was ist optimal?}
    \framesubtitle{Beschleunigung}
    \begin{definition}[Beschleunigung]
        \begin{equation}
            S_p(n) = \frac{T^\ast(n)}{T_p(n)} \leq p
        \end{equation}
    \end{definition}
    \begin{block}{optimale Beschleunigung $S_p(n) = \Theta(p)$}
        \begin{equation}
            p = \mathcal{O} \left( \frac{T^\ast(n)}{T_p(n)} \right)
        \end{equation}
    \end{block}
    
\end{frame}
%----------------------------------------------------------------------%SLIDE -

\subsection{Komplexitätsklassen}
%----------------------------------------------------------------------%SLIDE -
\begin{frame}
    \frametitle{Komplexitätsklassen}
    \begin{columns}
        \column{0.5\textwidth}
        \begin{figure}
            \centering
            % set diagram showing p vs. np
\begin{tikzpicture}
    [
    ]

    \node (p_set) at (0,0.25) [
        ellipse,
        dashed,
        draw,
        minimum height=4em,
        minimum width=6em,
    ] {};
    \node (np_set) at (0,1) [
        circle,
        draw,
        minimum height=9em,
    ] {};
    \node (nph_set) at (0,3) [
        circle,
        draw,
        minimum height=9em,
    ] {};
    \node (p) at (0,0.25) [
    ] {$P$};
    \node (np) at (-1.25,1) [
    ] {$NP$};
    \node (npc) at (0,2) [
    ] {$NP$-Complete};
    \node (nph) at (0,3.5) [
    ] {$NP$-Hard};

\end{tikzpicture}         

            \caption{$P$ vs. $NP$}
        \end{figure}
        \pause
        \column{0.5\textwidth}
        \begin{figure}
            \centering
            % set diagram showing nc vs. p
\begin{tikzpicture}
    [
    ]
    %\draw [help lines] (-2,-1) grid (2,5);

    \node (nc_set) at (0,0.25) [
        ellipse,
        dashed,
        draw,
        minimum height=4em,
        minimum width=6em,
    ] {};
    \node (p_set) at (0,1) [
        circle,
        draw,
        minimum height=9em,
    ] {};
    \node (ph_set) at (0,3) [
        circle,
        %draw,
        minimum height=9em,
    ] {};
    \node (nc) at (0,0.25) [
    ] {$NC$};
    \node (p) at (-1.25,1) [
    ] {$P$};
    \node (pc) at (0,2) [
    ] {$P$-Complete};
    \node (ph) at (0,3.5) [
    ] {$P$-Hard};

    \draw [domain=180:360] plot ({1.7*cos(\x)}, {3+1.7*sin(\x)});
    \draw (1.7,3) -- (1.7,3.6);
    \draw (-1.7,3) -- (-1.7,3.6);
    \draw [dotted] (1.7,3.6) -- (1.7,4.5);
    \draw [dotted] (-1.7,3.6) -- (-1.7,4.5);

\end{tikzpicture}         

            \caption{$NC$ vs. $P$}
        \end{figure}
    \end{columns}
\end{frame}
%----------------------------------------------------------------------%SLIDE -

%----------------------------------------------------------------------%SLIDE -
\begin{frame}
    \frametitle{Die Klasse $NC$}
    \begin{columns}
        \column{0.5\textwidth}
        \begin{definition}[$NC$]
            Die Menge aller Probleme, die parallel
            in $O(log^k n)$ Zeit auf $O(n^c)$ Prozessoren entschieden werden können.
        \end{definition}
        \begin{itemize}
            \item alle effizient parallel lösbaren Probleme
        \end{itemize}
        \column{0.5\textwidth}
        \begin{figure}
            \centering
            % set diagram showing nc vs. p
\begin{tikzpicture}
    [
    ]
    %\draw [help lines] (-2,-1) grid (2,5);

    \node (nc_set) at (0,0.25) [
        ellipse,
        dashed,
        draw,
        minimum height=4em,
        minimum width=6em,
    ] {};
    \node (p_set) at (0,1) [
        circle,
        draw,
        minimum height=9em,
    ] {};
    \node (ph_set) at (0,3) [
        circle,
        %draw,
        minimum height=9em,
    ] {};
    \node (nc) at (0,0.25) [
    ] {$NC$};
    \node (p) at (-1.25,1) [
    ] {$P$};
    \node (pc) at (0,2) [
    ] {$P$-Complete};
    \node (ph) at (0,3.5) [
    ] {$P$-Hard};

    \draw [domain=180:360] plot ({1.7*cos(\x)}, {3+1.7*sin(\x)});
    \draw (1.7,3) -- (1.7,3.6);
    \draw (-1.7,3) -- (-1.7,3.6);
    \draw [dotted] (1.7,3.6) -- (1.7,4.5);
    \draw [dotted] (-1.7,3.6) -- (-1.7,4.5);

\end{tikzpicture}         

            \caption{$NC$ vs. $P$}
        \end{figure}
    \end{columns}
\end{frame}
%----------------------------------------------------------------------%SLIDE -

%----------------------------------------------------------------------%SLIDE -
\begin{frame}
    \frametitle{$P$-vollständige Probleme}
    \framesubtitle{nicht effizient parallelisierbar}
    \begin{itemize}
        \item Generic Machine Simulation Problem (GMSP)
        \item Circuit Value Problem (CVP)
        \item Game Of Life (LIFE)
        \item Unifikation (UNIF)
    \end{itemize}
    unbekannt:
    \begin{itemize}
        \item Größter gemeinsamer Teiler (IntegerGCD)
        \item Erweiterter Euklidischer Algorithmus (ExtendedGCD)
        \item Diskrete Exponentialfunktion (ModPower)
    \end{itemize}
\end{frame}
%----------------------------------------------------------------------%SLIDE -
