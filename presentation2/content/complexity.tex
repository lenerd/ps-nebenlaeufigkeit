\section{Komplexität}

\subsection{Komplexitätsklassen}
%----------------------------------------------------------------------%SLIDE -
\begin{frame}
    \frametitle{Komplexitätsklassen}
    \begin{columns}
        \column{0.5\textwidth}
        \begin{figure}
            \centering
            % set diagram showing p vs. np
\begin{tikzpicture}
    [
    ]

    \node (p_set) at (0,0.25) [
        ellipse,
        dashed,
        draw,
        minimum height=4em,
        minimum width=6em,
    ] {};
    \node (np_set) at (0,1) [
        circle,
        draw,
        minimum height=9em,
    ] {};
    \node (nph_set) at (0,3) [
        circle,
        draw,
        minimum height=9em,
    ] {};
    \node (p) at (0,0.25) [
    ] {$P$};
    \node (np) at (-1.25,1) [
    ] {$NP$};
    \node (npc) at (0,2) [
    ] {$NP$-Complete};
    \node (nph) at (0,3.5) [
    ] {$NP$-Hard};

\end{tikzpicture}         

            \caption{$P$ vs. $NP$}
        \end{figure}
        \pause
        \column{0.5\textwidth}
        \begin{figure}
            \centering
            % set diagram showing nc vs. p
\begin{tikzpicture}
    [
    ]

    \node (nc_set) at (0,0.25) [
        ellipse,
        dashed,
        draw,
        minimum height=4em,
        minimum width=6em,
    ] {};
    \node (p_set) at (0,1) [
        circle,
        draw,
        minimum height=9em,
    ] {};
    \node (ph_set) at (0,3) [
        circle,
        draw,
        minimum height=9em,
    ] {};
    \node (nc) at (0,0.25) [
    ] {$NC$};
    \node (p) at (-1.25,1) [
    ] {$P$};
    \node (pc) at (0,2) [
    ] {$P$-Complete};
    \node (ph) at (0,3.5) [
    ] {$P$-Hard};

\end{tikzpicture}         

            \caption{$NC$ vs. $P$}
        \end{figure}
    \end{columns}
\end{frame}
%----------------------------------------------------------------------%SLIDE -

%----------------------------------------------------------------------%SLIDE -
\begin{frame}
    \frametitle{Die Klasse $NC$}
    \begin{columns}
        \column{0.5\textwidth}
        \begin{block}{$NC$}
            Die Menge aller Probleme, die parallel
            in $O(log^k n)$ Zeit auf $O(n^c)$ Prozessoren entschieden werden können.
        \end{block}
        \begin{itemize}
            \item alle effizient parallel lösbaren Probleme
        \end{itemize}
        \column{0.5\textwidth}
        \begin{figure}
            \centering
            % set diagram showing nc vs. p
\begin{tikzpicture}
    [
    ]

    \node (nc_set) at (0,0.25) [
        ellipse,
        dashed,
        draw,
        minimum height=4em,
        minimum width=6em,
    ] {};
    \node (p_set) at (0,1) [
        circle,
        draw,
        minimum height=9em,
    ] {};
    \node (ph_set) at (0,3) [
        circle,
        draw,
        minimum height=9em,
    ] {};
    \node (nc) at (0,0.25) [
    ] {$NC$};
    \node (p) at (-1.25,1) [
    ] {$P$};
    \node (pc) at (0,2) [
    ] {$P$-Complete};
    \node (ph) at (0,3.5) [
    ] {$P$-Hard};

\end{tikzpicture}         

            \caption{$NC$ vs. $P$}
        \end{figure}
    \end{columns}
\end{frame}
%----------------------------------------------------------------------%SLIDE -

%----------------------------------------------------------------------%SLIDE -
\begin{frame}
    \frametitle{$P$-vollständige Probleme}
    \begin{itemize}
        \item nicht effizient parallelisierbar
    \end{itemize}
\end{frame}
%----------------------------------------------------------------------%SLIDE -
